
\section{Calculatrice et arithmétique avec \sympy}

Pour les exercices suivants, il sera nécessaire d'importer les fonctions
suivantes:
\begin{verbatim}
from sympy import E,pi,I,oo,sqrt,cos,atan,log
from sympy import Rational,factorial,simplify,factorint
\end{verbatim}

\begin{exercice}
Calculer le numérateur et le dénominateur de la fraction 
$\displaystyle\frac{885046}{2570670}$ après simplification.
\end{exercice}

\begin{exercice}
Calculer le numérateur et le dénominateur du nombre rationnel égal à
\[
    \left(\frac{1}{678} + \frac{1}{134} + \frac{1}{904}\right)^{-1}
\]
\end{exercice}

\begin{exercice}
Calculer la partie réelle et imaginaire du nombre complexe
$\displaystyle\left(\frac{1+i\sqrt{3}}{1-i}\right)^{20}$.
\end{exercice}

\begin{exercice}
    Évaluer le nombre $e^{i\pi}+1$.
\end{exercice}

\begin{exercice}
    Trouver les diviseurs de 406628157024.
\end{exercice}

\begin{exercice}
Calculer 100 décimales de $\pi^e$.
%\verb|(pi**E).n(100)|
\end{exercice}

\begin{exercice}
Est-ce que le nombre $2\cos(\pi/5)$ est solution de l'équation $x^2-x-1=0$?
\end{exercice}

\begin{exercice}[Une formule due à Gauss]
    Calculer la valeur de
    \[
	\Theta = 12\arctan\frac{1}{38}+
	         20\arctan\frac{1}{57}+
	          7\arctan\frac{1}{239}+
	      24\arctan\frac{1}{268}.
    \]
    Quel est ce nombre (indice: calculer la valeur de $4\Theta$)?
    % Combien de décimales sont exactes?
\end{exercice}

\begin{exercice}
    Consulter la documentation de la fonction \texttt{factorial} de SymPy.
\end{exercice}

\begin{exercice}
    Avec combien de zéros le nombre $100!$ se termine-t-il?
\end{exercice}

\begin{exercice}
    Évaluer les expressions \texttt{oo + 1}, \texttt{3 ** oo},
    \texttt{log(oo)}, \texttt{oo + oo}, \texttt{oo - oo}. Quel nombre est
    représenté par \texttt{oo}?
\end{exercice}

\begin{exercice}[Utiliser \sympy en ligne]
    La page \url{http://live.sympy.org/} permet d'utiliser \sympy sur
    internet. Calculer l'intégrale $\int\sin(x)dx$ en écrivant
    \texttt{integrate(sin(x))} après les symboles \verb|>>>|. Appuyer sur
    \texttt{ENTRÉE} pour évaluer le résultat.
\end{exercice}

\begin{exercice}[Utiliser \sympy Gamma en ligne]
    La page \url{http://www.sympygamma.com/} permet d'utiliser \sympy
    sur internet. C'est une immitation de Wolfram Alpha.
    Tracer la fonction $\sin(x)$ sur l'intervalle $[0,\pi]$ en écrivant
    \texttt{plot(sin(x))} dans la cellule d'évaluation. Appuyer sur
    \texttt{ENTRÉE} pour évaluer le résultat ou sur la touche \texttt{=}.
    Tester quelques exemples aléatoires de la page d'accueil de \sympy Gamma.
\end{exercice}

