% \section*{Exercices du Chapitre 18: Autres structures de données}
\section*{Exercices du Chapitre 19: Tableaux et analyse de données avec Pandas}

Pour les exercices suivants, il sera nécessaire d'importer les fonctions
suivantes:
\begin{verbatim}
from math import log, sqrt
from sympy import Li
from pandas import Series, DataFrame, read_excel, read_csv
\end{verbatim}

\begin{exercice}
(Hypothèse de Riemann
\footnote{Mazur, Barry, et William Stein. Prime Numbers and the Riemann Hypothesis.
Cambridge University Press, 2015. \url{http://wstein.org/rh/}}).
\begin{itemize}
    \item Constuire un tableau de données de 1000 lignes ($0\leq
    x< 1000$) contenant les trois colonnes: 
    $\pi(x)$, le nombre de nombres premiers inférieurs ou égal à $x$;
    $x/\log x$, l'approximation de $\pi(x)$ faite par Gauss;
    ${\rm Li}(x)=\int_2^x \frac{dt}{\log t }$, l'approximation de $\pi(x)$
faite par Dirichlet.
    \item Ajouter une colonne dans le tableau qui donne la valeur absolue de la
différence entre $\pi(x)$ et ${\rm Li}(x)$. 
\end{itemize}
\begin{itemize}
\item Trouver (approximativement) une constante $C$ telle que
    $|\pi(x) - {\rm Li} (x)|$ vaut à peu près $C\cdot\left(\sqrt x \log x\right)$.
    % Answer: $\frac{1}{8\pi}$
\end{itemize}
\noindent\textbf{Remarque:}
Helge von Koch a montré en
1901\footnote{
Von Koch, Helge (1901). "Sur la distribution des nombres premiers". Acta
Mathematica 24 (1): 159–182.
\url{https://dx.doi.org/10.1007/BF02403071}} que l'Hypothèse de Riemann 
est vraie si et seulement si 
\[
    \pi(x) = {\rm Li} (x) +  O\left(\sqrt x \log x\right). 
\]
\end{exercice}

\begin{exercice}[Les arbres de Namur]
Télécharger le fichier \texttt{arbresremarquables.xls} au sujet des \emph{Arbres
remarquables Namur} du site \url{http://data.gov.be}. Créer un tableau de
données de Pandas à partir de ce fichier. Afficher les premières
et les dernières lignes du tableau. Quelle est la circonférence moyenne des
arbres remarquables de Namur? Quelle essence d'arbre est la plus représentée?
Est-ce que l'arbre le plus grand de Namur est situé dans un domaine privé ou
public? De quelle essence s'agit-il?
\end{exercice}

\begin{exercice}[Le capital au $21^e$ siècle, Tomas Pikkety]
Télécharger le fichier \texttt{capital21c\_tableauSI1.csv} sur la page du cours.
Créer un tableau de données de Pandas à partir de ce fichier. 
Utiliser ce tableau pour recréer le
graphique \emph{La part du décile supérieur dans le revenu total (y compris
plus-values) aux Etats-Unis, 1910-2010}
du chapitre d'introduction du livre de Piketty
\footnote{Thomas Piketty,
Le capital au 21e siècle,
Editions du Seuil, Septembre 2013, \url{http://piketty.pse.ens.fr/capital21c}}.
En observant le graphique, compléter les trous de la citation suivante:
\begin{quote}
\it
Lecture: la part du décile supérieur dans le revenu national américain est
passée de ..... \% dans les années 1910-1920 à moins de .....\% dans les années 1950
(il s'agit de la baisse mesurée par Kuznets); puis elle est remontée de moins de
.....\% dans les années 1970 à .....\% dans les années 2000-2010.
% Lecture: la part du décile supérieur dans le revenu national américain est
% passée de 45-50\% dans les années 1910-1920 à moins de 35\% dans les années 1950
% (il s'agit de la baisse mesurée par Kuznets); puis elle est remontée de moins de
% 35\% dans les années 1970 à 45-50\% dans les années 2000-2010.
\end{quote}
\end{exercice}


\begin{exercice}[La lotterie]
En novembre 2015, le site \emph{BuzzFeedNews} a fait des simulations pour savoir
quelles sont les chances de perdre de l'argent à la loterie. On peut consulter
leur jupyter notebook ici:
\begin{center}
\url{https://github.com/BuzzFeedNews/2015-11-lottery-simulations}
\end{center}
Faire la lecture du texte et du code de ce notebook Jupyter.
Quelles fonctionalités de la librairie \texttt{pandas} ont-ils utilisées?
Quelles sont les chances de perdre de l'argent en achetant des billets de
loterie selon les simulations effectuées par BuzzFeedNews?
\end{exercice}

\begin{exercice}[BuzzFeedNews]
Faîtes la lecture d'un autre sujet de votre choix parmi la liste de la page:
\begin{center}
\url{https://github.com/BuzzFeedNews/everything}
\end{center}
Dans le code associé à l'article, reconnaissez-vous des fonctions ou librairies
Python que vous connaissez? Trouver 3 choses que vous connaissez et que vous
avez appris dans ce cours. Trouver 3 choses que vous ne connaissez pas et que
vous n'avez pas appris dans ce cours.
\end{exercice}

\begin{exercice}[Nombre de naissances par jour]
Télécharger le fichier du 
\emph{Nombre de naissances par jour} pour la période du 1er janvier 2008 au 31
décembre 2014 à partir du site \url{http://data.gov.be}.
En moyenne, combien d'enfants naissent chaque jour en Belgique?
Entre 2008 et 2014, quelle année y a-t-il eu le plus de naissance le jour du 10
mai? Y a-t-il une grande différence entre le minimum et le maximum pour le jour
du 10 mai? Plus difficile: quel mois a vu naître le plus d'enfants pendant ces
sept années?
\end{exercice}

\begin{exercice}[\texttt{data.gov.be}]
Télécharger un fichier de données de votre choix parmi les 4800+ jeux de données
disponibles sur \url{http://data.gov.be}. Tenter de l'ouvrir dans pandas (il y a
parfois des erreurs d'importation avec le format excel, les données sont
rarement parfaites dans la vraie vie. Dans ce cas, une option est de sauvegarder
la feuille excel sous le format csv à partir d'Excel ou Libre Office).
Réfléchissez à une question de votre choix au sujet de ces données. Répondez à
votre question.
\end{exercice}


