
\section*{Exercices du Chapitre 10: Geogebra}


\begin{exercice}
En créant quelques points, des cercles et des segments, dessiner une voiture.
\end{exercice}

\begin{exercice}
La page \url{http://tube.geogebra.org/student/m60309} montre six carrés
construits de différentes manières.  Examinez chacun des carrés en déplaçant
leurs sommets à la souris. Découvrez quels quadrilatères restent des carrés lors
du déplacement de ses sommets. Essayez d'établir une conjecture sur la manière
dont chacun des quadrilatères a été construit.
\end{exercice}

Dans les trois prochains exercices, on cherche à construire des polygones
réguliers qui satisfont au \emph{test du déplacement}, c'est-à-dire que le
déplacement de ses sommets préserve la propriété d'être un polygone régulier.

% voir http://www.geogebra.org/wiki/en/Tutorial:Practice_Block_I
% pour les etapes
\begin{exercice}
Construire un triangle équilatéral. 
\end{exercice}

\begin{exercice}
Construire un hexagone régulier.
\end{exercice}

% \begin{exercice}
% Construire un cercle circonscrit à un triangle. (donner les étapes)
% \end{exercice}

% \begin{exercice}
% Visualiser le théorème de Thales. (donner les étapes)
% \end{exercice}

\begin{exercice}
Construire un pentagone régulier en suivant les étapes
suivantes:
%\footnote{\url{https://fr.wikipedia.org/wiki/Construction_du_pentagone_régulier_à_la_règle_et_au_compas}}:
\begin{enumerate}
\item    Tracer un cercle $\Gamma$ de centre $O$ et de rayon $R$ (unité quelconque).
\item    Tracer deux diamètres perpendiculaires, $[AC]$ et $[BD]$.
\item    Tracer le milieu $I$ de $[OA]$.
\item    Tracer le cercle $\Gamma'$ de centre $I$ et passant par $O$ de rayon $R/2$. $\Gamma'$ passe donc aussi en $A$.
\item    Tracer une droite passant par $B$ et $I$ intersectant $\Gamma'$ en $E$ et $F$ ($E$ est plus proche de $B$).
\item    Tracer deux (arcs de) cercles $\Gamma_1$ et $\Gamma_2$ de centre $B$ et de rayons
(respectivement) $BE$ et $BF$.  $\Gamma_1$ et $\Gamma_2$ intersectent $\Gamma$
en quatre points $D_1$, $D_2$, $D_3$, $D_4$.
\item Construire le polygone qui passe par $D$, $D_1$, $D_2$, $D_3$,
$D_4$ et afficher ses angles internes.
\item Nettoyer en cachant les éléments intermédiaires ayant permis la
construction.
%\item Faire le test du déplacement.
\end{enumerate}
\end{exercice}
% https://fr.wikipedia.org/wiki/Construction_du_pentagone_r%C3%A9gulier_%C3%A0_la_r%C3%A8gle_et_au_compas#Pentagone_inscrit_dans_un_cercle

\begin{exercice}
Vérifier avec GeoGebra l'exercice III.5.20 du Chapitre III Géométrie affine du
\emph{Cahier d'exercices de géométrie élémentaire} de Pierre Lecomte et Michel
Rigo.
\end{exercice}

% \url{https://en.wikipedia.org/wiki/Desargues's_theorem}
% \url{http://2000clicks.com/mathhelp/GeometryTheorem-Summary.aspx#tria}
% \url{http://2000clicks.com/mathhelp/GeometryTriangleCyclicQuadrilateralButterflyTheorem.aspx}

\begin{exercice}[Théorème du papillon]
%\url{https://fr.wikipedia.org/wiki/Th%C3%A9or%C3%A8me_du_papillon}
Tracer un cercle. Définir le point milieu $M$ d'une corde $[PQ]$. Construire deux
autres cordes $[AB]$ et $[CD]$ passant par $M$ de sorte que les points $A$ et
$C$ soient sur le même arc de cercle de $P$ à $Q$. Définir les points
d'intersection $X$ et $Y$ entre les cordes 
$[AB]$ et $[CD]$ respectivement et la corde $[PQ]$.
Définir le point milieu $M'$ du segment $[XY]$.
Tester que $M=M'$ en déplaçant les points indépendants.
Utiliser l'outil $a=?b$ pour tester que $M$ est bien égal à $M'$ en les
sélectionnant dans la fenêtre Algèbre si nécessaire.
\end{exercice}

Pour les deux prochains exercices, on suggère de \emph{cacher} les objets
intermédiaires ayant permis la construction des objets désirés. Attention, il ne
s'agit pas de les effacer.

\begin{exercice}%[Cercle d'Euler]
Dans un triangle quelconque, quelle relation y a-t-il entre les milieux des
côtés et les pieds des hauteurs?
\end{exercice}

% https://en.wikipedia.org/wiki/Triangle_center
% https://fr.wikipedia.org/wiki/Triangle
% https://fr.wikipedia.org/wiki/Droite_d'Euler
% https://fr.wikipedia.org/wiki/%C3%89l%C3%A9ments_remarquables_d'un_triangle
\begin{exercice}%[Droite d'Euler]
Dans un triangle quelconque, quelle relation y a-t-il entre l'orthocentre
(intersection des hauteurs), le centre de gravité (intersection des médianes) et
le centre du cercle circonscrit?
\end{exercice}

\begin{exercice}
Utiliser les ressources de GeoGebra sur internet.
\begin{itemize}
\item Regarder une vidéo de votre choix de
\url{https://www.youtube.com/user/GeoGebraChannel}
\item Tester une feuille de travail de votre choix parmi les plus de 300 000
disponibles sur \url{http://tube.geogebra.org/}
\item Tester l'utilisation de GeoGebra sur internet à l'adresse
\url{https://app.geogebra.org/}
\end{itemize}
\end{exercice}

