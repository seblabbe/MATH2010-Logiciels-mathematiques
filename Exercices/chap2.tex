
\section*{Exercices du Chapitre 2: Calculatrice et arithmétique avec \python}

Pour les exercices suivants, il sera nécessaire d'importer les fonctions
suivantes:
\begin{verbatim}
from math import e,pi,sqrt,sin,cos,tan,asin,acos,atan,log
\end{verbatim}

\begin{exercice}
Calculer le nombre $2^5\cdot 3^2\cdot 7$.
\end{exercice}

\begin{exercice}
Calculer le résultat de $540931 \div 83$ sous forme décimale.
\end{exercice}

\begin{exercice}
Calculer le quotient et le reste de la division de 540931 par 83.
\end{exercice}

\begin{exercice}
Calculer la somme et la moyenne des nombres suivants: 
\[
    3121, 71072, 44563, 10986, 43048, 8081, 44681, 23757, 4093, 24805, 30750,
    73504.
\]
\end{exercice}

\begin{exercice}
    Quelle opération est effectuée en premier dans l'expression
    \texttt{7 * 2 ** 5 + 3} ? Laquelle est effectuée en dernier?
    Est-ce que la valeur de l'expression est égale à $7\cdot 2^{5+3}$?
\end{exercice}

\begin{exercice}
    Soit l'expression \texttt{3 ** 3 ** 3}. 
    Quelle opération d'exponentiation (\texttt{**}) est effectuée en premier :
    la première ou la deuxième?
\end{exercice}

\begin{exercice}
    Soit $p(x)= -21x^{5} - 10x^{4} + x^{3} + x^{2} - 1$. Évaluer $p(x)$ pour
    toutes les valeurs de $x$ dans l'ensemble $\{6, 7, 9, 13, 16, 17, 19\}$.
\end{exercice}

\begin{exercice}
    Résoudre l'équation $(x+1)^7=62748517$.
\end{exercice}

\begin{exercice}
    Calculer l'aire de la surface d'un cube dont le volume est 
    $4913\,\mathrm{m}^3$.
\end{exercice}

\begin{exercice}
    Est-ce qu'une sphère dont le diamètre est $12$ cm peut contenir 1 litre
    d'eau?
\end{exercice}

\begin{exercice}
Résoudre l'équation $12^a=8916100448256$.
\end{exercice}

\begin{exercice}
Calculer 10 décimales de $\pi^e$.
%\verb|(pi**E).n(100)|
\end{exercice}

\begin{exercice}
    Calculer les angles intérieurs d'un triangle dont les côtés
    mesurent 10 cm, 24 cm et 26 cm.
\end{exercice}

\begin{exercice}
    Quel est le périmètre d'un triangle ayant des angles intérieurs de 20 et
    50 degrés et dont le plus long côté mesure 10 cm.
\end{exercice}

% \reflectbox{Réponses: 
% 3-2016,
% 4-2016,
% 3-2016, }
%\rotatebox{180}{Perlmonks}


