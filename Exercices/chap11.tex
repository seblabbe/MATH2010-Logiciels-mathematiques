
\section*{Exercices du Chapitre 11: Types de données de Python}

\begin{exercice}
    Vrai ou faux.
    En Python 3, toutes les opérations de base sur les entiers de type
    \texttt{int} comme 
    \texttt{4 + 7},
    \texttt{4 * 7},
    \texttt{4 - 7},
    \texttt{4 / 7}
    retournent un objet de type \texttt{int}.
\end{exercice}

\begin{exercice}
    Quelle est la différence, s'il y en a une, entre les objets de la liste
    \texttt{[4, 4., '4.0', "4.0", 4.0]}?
\end{exercice}

\begin{exercice}
    Est-ce que tout objet \texttt{X} de type \texttt{int} peut être
    transformé en un objet de type \texttt{str} avec la fonction
    \texttt{str(X)}?
    Est-ce que tout objet \texttt{X} de type \texttt{str} peut être
    transformé en un objet de type \texttt{int} avec la fonction
    \texttt{int(X)}?
\end{exercice}

\begin{exercice}
    Trouver une opération binaire \texttt{op} 
    telle que \texttt{4 op 7} retourne un objet de type \texttt{bool}.
\end{exercice}

\begin{exercice}
    Trouver toutes les solutions $\texttt{X},\texttt{Y}\in\{\texttt{True},
    \texttt{False}\}$ de l'équation booléenne suivante: 
    \begin{center}
	\texttt{not (not X and not Y) and not (X and Y) == True}
    \end{center}
    Peut-on simplifier cette équation?
\end{exercice}

\begin{exercice}
    Soit la chaîne de caractères \texttt{w='logiciels mathematiques'}. 
Pour quelles valeurs entières positives et négatives de \texttt{a} est-ce que
\texttt{w[a]=='i'}?
Pour quelles valeurs entières positives de \texttt{a} et \texttt{b} est-ce que
\texttt{w[a:b]} vaut \texttt{'ciels math'}?
\end{exercice}


\begin{exercice}
Soit la chaîne de caractère
\begin{center}
\texttt{a = 'lundi mardi mercredi jeudi vendredi samedi dimanche'}.
\end{center}
En utilisant les méthodes des objets de type \texttt{str} (écrire \texttt{a.} et
appuyer sur la touche tabulations pour afficher les méthodes), trouver une
façon de créer les chaînes de caractères et listes suivantes:
\begin{center}
\begin{tabular}{l}
\texttt{'LUNDI MARDI MERCREDI JEUDI VENDREDI SAMEDI DIMANCHE'}\\
\texttt{'Lundi mardi mercredi jeudi vendredi samedi dimanche'}\\
\texttt{'lundx mardx mercredx jeudx vendredx samedx dxmanche'}\\
\texttt{['lundi', 'mardi', 'mercredi', 'jeudi', 'vendredi', 'samedi', 'dimanche']}
\end{tabular}
\end{center}
\end{exercice}

% sage: a = 'lundi mardi mercredi jeudi vendredi samedi dimanche'
% sage: a.split()
% ['lundi', 'mardi', 'mercredi', 'jeudi', 'vendredi', 'samedi', 'dimanche']
% sage: a.upper()
% 'LUNDI MARDI MERCREDI JEUDI VENDREDI SAMEDI DIMANCHE'
% sage: a.replace(' ', '+')
% 'lundi+mardi+mercredi+jeudi+vendredi+samedi+dimanche'
% sage: a.capitalize()
% 'Lundi mardi mercredi jeudi vendredi samedi dimanche'


