\documentclass[12pt,a4paper]{article}

\usepackage{amsmath}
\usepackage{amsfonts}
\usepackage{latexsym}
\usepackage{graphicx}
\usepackage{amssymb}
\usepackage{amsthm}
\usepackage[margin=2cm,includehead]{geometry}
\usepackage[T1]{fontenc}     % Hyphénation, copier-coller des mots accentués
\usepackage{lmodern}         % Polices vectorielles
\usepackage[french]{babel}   % Vocabulaire en francais
\usepackage[utf8]{inputenc}  % Codage UNICODE (UTF-8)
\usepackage{url}
\usepackage{color}
\usepackage{enumerate}
\usepackage[shortlabels]{enumitem} %pour commencer les enumerations a des nombres differents
\usepackage[pdftex,pagebackref]{hyperref}
\usepackage{cleveref}
\usepackage{titlesec}  % for using titleformat
\usepackage{rotating}

%\usepackage[disable]{todonotes}
\usepackage{todonotes}

% Page foot 
\usepackage{fancyhdr}
\pagestyle{fancy}
\lhead{\sffamily\bfseries\small Exercices pour les TP}
\rhead{\sffamily\bfseries\small MATH2010-1 Logiciels mathématiques}
%\renewcommand{\headrulewidth}{0.4pt}% remove header line
\renewcommand{\footrulewidth}{0pt}% default is 0pt
%\cfoot{}

% ceci immite le format par defaut des titres de section de la classe scrartcl
\titleformat{\section}
  {\normalfont\sffamily\large\bfseries\centering\color{black}}
  {\thesection}{0.5em}{}

% entier, réels, etc
\newcommand{\N}{\mathbb{N}}
\newcommand{\Z}{\mathbb{Z}}
\newcommand{\Q}{\mathbb{Q}}
\newcommand{\R}{\mathbb{R}}
\newcommand{\sympy}{\texttt{SymPy} }
\newcommand{\python}{\texttt{Python} }

% others
\newcommand{\bw}{\mathbf{w}}
\renewcommand{\a}{{\vec{a}}}

% Blocks
\theoremstyle{definition}
\newtheorem{exercice}{Exercice}[section]
\newtheorem{exemple}[exercice]{Exemple}
\newtheorem{remarque}[exercice]{Remarque}
\newtheorem{question}[exercice]{Question}




\begin{document}

\section*{Devoir 1}

En utilisant SymPy ou Mathematica mais un seul des deux, répondre aux
questions suivantes.

Écrire vos réponses dans le fichier \texttt{Devoir-1-<MATRICULE>-<NOM>.ipynb}
ou\\ \texttt{Devoir-1-<MATRICULE>-<NOM>.nb} selon que vous utilisiez SymPy ou
Mathematica téléchargeable sur la page web du cours. Inclure la démarche, les
réponses aux questions et les justifications. Envoyer votre fichier avant le
\textbf{jeudi 24 mars à 23h59} par courriel à l'adresse
\texttt{slabbe@ulg.ac.be} en remplaçant \texttt{<MATRICULE>} par votre numéro
de matricule et \texttt{<NOM>} par votre nom.

Pour écrire du texte entre les cellules et justifier une réponse, utiliser
\texttt{Format > Style > Text} en Mathematica, et \texttt{Cell > Cell Type >
Markdown} en SymPy et Jupyter.

Le plagiat sera détecté et entraînera une note de zéro pour les personnes
impliquées.

\begin{question}
Le Théorème de
Gauss--Wantzel\footnote{\url{https://fr.wikipedia.org/wiki/Théorème_de_Gauss-Wantzel}} dit qu'un polygone régulier à $n$ côtés est
constructible avec la règle et le compas si et seulement si $n$ est le produit
d'une puissance de $2$ et de nombres premiers de Fermat distincts dont les seuls
connus sont 3, 5, 17, 257 et 65537.
Est-ce qu'un polygone régulier à 6 côtés est constructible?
Est-ce qu'un polygone régulier à 24480 côtés est constructible?
Est-ce qu'un polygone régulier à 88305875025920 côtés est constructible?
\end{question}

\begin{question}
Résoudre l'équation $x^3-3x^2-5=0$ et donner une valeur numérique approchée des
solutions.
\end{question}

\begin{question}
Tracer la surface de Dini 
\footnote{\url{https://en.wikipedia.org/wiki/Dini's_surface}}
dont les équations paramétriques sont:
\begin{align*}
    x&=a\cos\left(u\right)\sin\left(v\right)\\
    y&=a\sin\left(u\right)\sin\left(v\right)\\
    z&=a\left(\cos\left(v\right)+\ln\left(\tan\left(\frac{v}{2}\right)\right)\right)+bu 
\end{align*}
pour $a=1$, $b=1$ sur les intervalles $0\leq u \leq 5\pi$ et $0.01 \leq v\leq
1$.
% In [64]: from sympy.abc import u,v
% In [65]: x = cos(u)*sin(v)
% In [66]: y = sin(u)*sin(v)
% In [67]: z = (cos(v)+log(tan(v/2)))+2*u
% In [68]: from sympy.plotting import plot3d_parametric_surface
% In [69]: plot3d_parametric_surface(x,y,z, (u,0,5*pi), (v,0.1,1)).save('yo.png')
\end{question}


% from http://mathematica.stackexchange.com/users/9490/jason-b
% \begin{verbatim}
% a = 2;
% f = (a + Cos[u/2] Sin[t] - Sin[u/2] Sin[2 t]) Cos[u];
% g = (a + Cos[u/2] Sin[t] - Sin[u/2] Sin[2 t]) Sin[u];
% h = Sin[u/2] Sin[t] + Cos[u/2] Sin[2 t];
% ParametricPlot3D[{
%   {f, g, h},
%   {h, (f - a), g}},
%  {t, 0, 2 Pi}, {u, 0, 2 Pi}, ColorFunction -> (Hue[#5] &),
%  Mesh -> None,
%  Boxed -> False,
%  Axes -> False,
%  PlotPoints -> {60, 60},
%  ViewPoint -> {2, 1, -2.5},
%  ViewVertical -> {2., 0.2, 0.0}
%  ]
% \end{verbatim}

\begin{question}
Soit $p(x)=- x^{4} + 28 x^{3} - 221 x^{2} + 350 x + 600$ un polynôme. Trouver
l'ensemble des valeurs de $x$ telles que $p(x)$ atteint un optimum local. Dire
s'il s'agit d'un minimum ou un maximum. Calculer l'aire de la région
$A=\{(x,y):0\leq y\leq p(x)\}$ bornée supérieurement par le polynôme $p(x)$ et
inférieurement par l'abscisse.
\end{question}

\begin{question}
On considère la fonction $f(x)=x^{x\over 1-x}$ pour tout réel $x>0$. Donner le
domaine de définition de $f$. Calculer les limites de la fonction $f$ aux bornes
des intervalles qui composent le domaine de $f$. Expliquez comment prolonger $f$
par continuité aux points $x=0$ et $x=1$.
\end{question}

% \begin{question}
% %http://math.univ-lille1.fr/~tumpach/Site/Enseignement_files/interro2-groupe11.pdf
% Résoudre, suivant les valeurs du paramètre $a$, le système
% \[
% \left\{
% \begin{array}{lll}
% ax    & +2y&=a+1\\
% (a-1)x& -ay&=2a
% \end{array}\right.
% \]
% \end{question}

\begin{question}
On considère les vecteurs
$v_1=(-8, 1, -10, 1, 6)$,
$v_2=(-6, -10, 2, 10, -3)$,
$v_3=(-2, 8, 10, 1, 10)$,
$v_4=(-14, -9, -8, 11, 3)$,
$v_5=(-2, -3, 5, -8, -6)$.
Donner une base du sous espace vectoriel engendré par $v_1$, $v_2$, $v_3$, $v_4$
et $v_5$.
Le vecteur $w=(0, -6, -1, -8, 10)$
est-il dans ce sous-espace vectoriel? Si oui, l'exprimer  comme combinaison
linéaire des vecteurs de la base.
\end{question}

\end{document}

