
\begin{exercice}
Apprendre quelques raccourcis clavier de Jupyter.
Consulter \texttt{Help > Keyboard Shortcuts} au besoin.
\begin{itemize}
\item Qu'est-ce qui caractérise le mode édition (3 réponses), le mode commande
(3 réponses)?
\item Quelle touche du clavier permet de passer du mode édition au mode commande
et du mode commande au mode édition?
\item En mode commande, les fléches $\uparrow$ et $\downarrow$ permettent de
sélectionner la cellule au-dessus et au-dessous respectivement. Quelles touches
alphabétiques du clavier permettent de faire la même chose. Pourquoi ce sont ces
touches-là?
\item En mode commande, quelle touche permet d'ajouter une nouvelle cellule
au-dessus de la cellule sélectionnée? au-dessous? Pourquoi ce sont ces touches-là?
\item En mode commande, quelles touches permettent de couper, de copier et de
coller une cellule? Pourquoi ce sont ces touches-là?
\item En mode édition, quelle touche permet de compléter les mots ou le code que
l'on écrit?
\end{itemize}
\end{exercice}

\section{Calcul symbolique}

% Read this \url{https://en.wikipedia.org/wiki/Pi#Polygon_approximation_era}

Pour les exercices suivants, il sera nécessaire d'importer les fonctions
et symboles suivants:
\begin{verbatim}
from sympy import expand,factor,simplify,sin,tan,sec,radsimp,ratsimp,apart,collect
from sympy.abc import a,b,c,d,n,x,y,z
\end{verbatim}

\begin{exercice}
Développer $(x+y)^2(x-y)(x^2+y)$.
%\verb|((x + y)**2 * (x - y) * (x**2 + y)).expand()|
\end{exercice}

\begin{exercice}
    Calculer le coefficient de $x^3y^7z^2$ dans l'expression $(x+y+z)^{12}$.
% 7920
\end{exercice}

\begin{exercice}
Simplifier $\sec(x)^2-1$.
\end{exercice}

% \begin{exercice}
% Simplifier $\cos(7x)-\sin(7x)$.
% \end{exercice}

\begin{exercice}
Simplifier 
$\displaystyle
\frac{\sqrt{a} - \sqrt{b}}{\sqrt{a} + \sqrt{b}} + 
\frac{\sqrt{a} + \sqrt{b}}{\sqrt{a} - \sqrt{b}}$.
\end{exercice}

\begin{exercice}
Factoriser le polynôme $25 x^{5} + 175 x^{4} - 40 x^{3} - 280 x^{2} + 16 x + 112$.
\end{exercice}

\begin{exercice}
Montrer que $\cos^3(x)+\sin^3(x)$ peut s'écrire sous la forme
\[
(\cos(x)+\sin(x)) (1-\frac{1}{2}\sin(2x)).
\]
\end{exercice}

\begin{exercice}
Rationaliser le dénominateur de $\displaystyle\frac{2 + \sqrt{7}}{3 - \sqrt{13}}$.
\end{exercice}

\begin{exercice}
Calculer le coefficient de $x^2$ dans l'expression $(x-a)(x-b)(x-c)(x-d)$.
\end{exercice}

\begin{exercice}
Décomposer
$
\displaystyle
\frac{x^{3} - 10 x^{2} + 40 x - 47}{x^{4} - 9 x^{3} + 17 x^{2} + 33 x - 90}
%= \frac{1}{x + 2} - \frac{1}{x - 3} - \frac{1}{\left(x - 3\right)^{2}} + \frac{1}{x - 5}
$
en somme de fractions rationnelles.
\end{exercice}

\begin{exercice}
    Écrire l'expression
$\displaystyle\frac{5}{x + 3} + \frac{3}{x - 5} - \frac{8}{x}$
sous un dénominateur commun.
\end{exercice}

\begin{exercice}
Factoriser le polynôme $x^k-1$ en produit de polynômes irréductibles sur $\Q$
lorsque $k=2,3,4,5,6,7$.  Pour quelles valeurs entières de $k\geq2$ est-ce que
$\displaystyle\frac{x^k-1}{x-1}$ est-il égal à un polynôme irréductible sur
$\Q$?
\end{exercice}

\begin{exercice}
    Trouver l'expression symbolique qui calcule le volume d'un cube de côté
    $2a$ auquel on a soustrait le volume d'une boule de rayon $a$ et de même
    centre.
\end{exercice}

\begin{exercice}\label{exo:apotheme}
Soit un polygone régulier à $n$ côtés dont la longueur des côtés est $a$.
Définir dans Sympy l'expression symbolique qui donne l'apothème
\footnote{L'\emph{apothème} est la longueur de la perpendiculaire menée du
centre vers un côté d'un polygone régulier.} en fonction de $n$ et de $a$. 
\end{exercice}


\begin{exercice}
En procédant par substitution sur l'expression symbolique obtenue à
l'exercice précédent, calculer l'apothème d'un hexagone régulier de
côté $3$ cm et l'apothème d'un heptadécagone régulier de côté $89$ km.
\end{exercice}

\begin{exercice}
    Trouver l'expression $R(n)$ qui donne le rapport entre le périmètre d'un
    polygone régulier à $n$ côté et le double de son apothème. 
    Évaluer $R(n)$ pour des valeurs de plus en plus grandes de $n\geq3$.
    En déduire le comportement de $R(n)$ lorsque $n$ tend vers l'infini.
    \textit{Facultatif}: En utilisant la commande \texttt{plot} décrite dans le
    début du chapitre 6, tracer le graphique de cette expression en fonction de $n$
    pour $n\in[3,30]$. 
\end{exercice}

