
\section{Algèbre linéaire}

Pour les exercices suivants, il sera nécessaire d'importer les fonctions
et symboles suivants:
\begin{verbatim}
from sympy import Matrix,randMatrix,simplify,I
from sympy.abc import m
\end{verbatim}


\begin{exercice}
    En utilisant \texttt{randMatrix}, construire deux matrices carrées 
$A$ et $B$ dont les coefficients sont choisis aléatoirement. Vérifier que
$A+B=B+A$, $(AB)^{-1}=B^{-1}A^{-1}$ et (très probablement) $AB\neq BA$.
\end{exercice}

\begin{exercice}
Soit la matrice
\[
\frac{\sqrt{2}}{2}
\left[\begin{matrix}1 & i\\i & 1\end{matrix}\right]
\]
Calculer $M^9$ et le déterminant de $M$. Sans utiliser l'ordinateur, en déduire
que $M^8=I$ et $M^{-1}=M^7$ et deviner ce que vaut $M^{2015}$. Vérifier ensuite
les résultats avec l'ordinateur.
\end{exercice}


\begin{exercice}
    Dans le graphe ci-bas, entre le sommet $0$ et le sommet $4$, il y a 
    un chemin de longueur $1$ ($0\to4$),
    deux chemins de longueur $2$ ($0\to3\to4$, $0\to2\to4$) et 
    six chemins de longeurs $3$ 
    ($0\to2\to0\to4$, $0\to3\to0\to4$, $0\to4\to0\to4$, $0\to4\to1\to4$, 
    $0\to4\to2\to4$, $0\to4\to3\to4$).  
    Calculer le
    nombre de chemins de longueur $L$ du sommet $0$ au sommet $4$ pour toutes
    les valeurs de $L\in\{4,5,6,50\}$.
\begin{center}
\begin{tikzpicture}
\node (0) at (0,0) {$0$};
\node (4) at (2,0) {$4$};
\node (3) at (1,1) {$3$};
\node (2) at (1,-1) {$2$};
\node (1) at (3.5,0) {$1$};
\draw (0) -- (4) -- (1);
\draw (0) -- (3) -- (4) -- (2) -- (0);
\end{tikzpicture}
\end{center}
% >>> L = [0, 0, 1, 1, 1, 0, 0, 0, 0, 1, 1, 0, 0, 0, 1, 1, 0, 0, 0, 1, 1, 1, 1, 1, 0]
% >>> M = Matrix(5,5,L)
% >>> M**50
\end{exercice}

\begin{exercice}
    En calculant la forme échelonnée
    réduite\footnote{\url{https://fr.wikipedia.org/wiki/Matrice_échelonnée}}
    d'une matrice, résoudre les systèmes d'équations linéaires $AX=B$
    suivants:
\[
A = \left[\begin{matrix}3 & 2 & -1\\2 & -3 & 4\\1 & 1 & -3\end{matrix}\right],
B = \left[\begin{matrix}1\\3\\2\end{matrix}\right];
\qquad
A = \left[\begin{matrix}3 & 2 & 1\\2 & -3 & 5\\1 & 1 & 0\end{matrix}\right],
B = \left[\begin{matrix}1\\-8\\1\end{matrix}\right].
\]
\end{exercice}


\begin{exercice}
Résoudre le système linéaire $AX=B$ suivant à un paramètre $m$. On pourra
distinguer plusieurs cas selon les valeurs de $m$.
\[
A = \left[\begin{matrix}1 & 1 & m\\1 & 1 & -1\\1 & m & -1\end{matrix}\right],
B = \left[\begin{matrix}m\\1\\1\end{matrix}\right].
\]
% >>> from sympy.abc import m
% >>> M = Matrix(3,4,[1,1,m,m,1,1,-1,1,1,m,-1,1])
% >>> simplify(M.rref()[0])
% [          2*m ]
% [1  0  0  -----]
% [         m + 1]
% [              ]
% [0  1  0    0  ]
% [              ]
% [         m - 1]
% [0  0  1  -----]
% [         m + 1]
\end{exercice}

\begin{exercice}
Dire si les familles de vecteurs ci-bas sont linéairement indépendants:
\[
\left\{
\left(\begin{matrix}5\\3\\-2\end{matrix}\right),
\left(\begin{matrix}8\\1\\3\end{matrix}\right)
\right\},\qquad
\left\{
\left(\begin{matrix}5\\3\\-2\end{matrix}\right),
\left(\begin{matrix}8\\1\\3\end{matrix}\right),
\left(\begin{matrix}7\\-11\\22\end{matrix}\right)
\right\}.
\]
\end{exercice}


\begin{exercice}
    Calculer le volume du paralépipède de $\R^3$ dont les sommets sont:
\[
\left(\begin{matrix}0\\0\\0\end{matrix}\right),
\left(\begin{matrix}45\\79\\89\end{matrix}\right),
\left(\begin{matrix}74\\26\\39\end{matrix}\right),
\left(\begin{matrix}70\\71\\91\end{matrix}\right),
\left(\begin{matrix}119\\105\\128\end{matrix}\right),
\left(\begin{matrix}115\\150\\180\end{matrix}\right),
\left(\begin{matrix}144\\97\\130\end{matrix}\right),
\left(\begin{matrix}189\\176\\219\end{matrix}\right).
\]
% >>> L = [45, 79, 89, 74, 26, 39, 70, 71, 91]
% >>> Matrix(3,3,L).det()
% -28825
\end{exercice}

\begin{exercice}
Soit la matrice $M$ et le vecteur $v$ suivants:
\[
M=
\left[\begin{matrix}10 & \frac{7}{3} & -7\\9 & 10 & -9\\3 & \frac{7}{3} &
0\end{matrix}\right],\qquad
v=
\left[\begin{matrix}1\\3\\1\end{matrix}\right].
\]
Calculer $Mv$. Déduire la propriété de ce vecteur. Trouver tous les vecteurs
possédant la même propriété.
\end{exercice}


