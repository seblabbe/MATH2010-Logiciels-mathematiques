

\section*{Exercices du Chapitre 12: Listes}

\begin{exercice}
Définir les trois listes
\texttt{A=['lundi', 'mardi']},
\texttt{B=['mercredi', 'jeudi']} et
\texttt{C=['vendredi', 'samedi', 'dimanche']}.
En utilisant \texttt{A}, \texttt{B} et \texttt{C},
construire la liste des sept jours de la semaine à laquelle on a enlevé mercredi.
Construire la semaine des quatre jeudis.
Construire la liste des jours ouvrables de la semaine. 
Trier cette dernière liste en ordre lexicographique décroissant.
\end{exercice}

\begin{exercice}
Construire la liste des carrés et des cubes de tous les nombres de 20 à 40.
\end{exercice}

\begin{exercice}
Construire la liste des puissances de 5 pour toutes les valeurs d'exposant $a$
dans l'intervalle $[6,14]$.
\end{exercice}

\begin{exercice}
Construire la liste des sinus des angles de 0 à 90 degrés par pas de 5 degrés.
\end{exercice}

\begin{exercice}
Voici une liste décrivant la taille de 64 fichiers:
{\footnotesize
\begin{verbatim}
L = ["12K", "12K", "12K", "12K", "12K", "12K", "12K", "12K", "16K", "16K", "16K", "20K", "20K", 
"20K", "24K", "24K", "24K", "24K", "24K", "24K", "24K", "24K", "24K", "28K", "32K", "40K", "4K", 
"4K", "4K", "4K", "4K", "4K", "4K", "4K", "4K", "4K", "4K", "4K", "4K", "4K", "4K", "4K", "4K", 
"4K", "4K", "4K", "4K", "4K", "4K", "4K", "4K", "4K", "4K", "4K", "4K", "4K", "4K", "4K", "8K", 
"8K", "8K", "8K", "8K", "8K"] 
\end{verbatim}
}
Construire cette liste \texttt{L} et s'en servir pour répondre aux queestions
suivantes. Calculer la taille totale des fichiers,
leur moyenne et la médiane.
Combien de fichiers y a-t-il de fichiers de chaque taille?
\end{exercice}


