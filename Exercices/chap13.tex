
\section*{Exercices du Chapitre 13: Boucle \texttt{for}}

% Les exercices ci-bas sont tirés du livre de Gérard Swinnen 
% \footnote{Gérard Swinnen, Apprendre à programmer avec Python 3, 2012.
% \url{http://inforef.be/swi/download/apprendre_python3_5.pdf}}
% que vous êtes invités à consulter pour en apprendre d'avantage.

\begin{exercice}
Soient les listes suivantes :
\begin{verbatim}
    t1 = [31, 28, 31, 30, 31, 30, 31, 31, 30, 31, 30, 31]
    t2 = ['Janvier', 'Février', 'Mars', 'Avril', 'Mai', 'Juin',
    'Juillet', 'Août', 'Septembre', 'Octobre', 'Novembre', 'Décembre']
\end{verbatim}
En utilisant \texttt{t1} et \texttt{t2}
créer une nouvelle liste \texttt{t3} qui contient
tous les éléments des deux listes en les alternant, de telle manière
que chaque nom de mois soit suivi du nombre de jours correspondant :
\texttt{['Janvier',31,'Février',28,'Mars',31, etc...]}.
\end{exercice}

\begin{exercice}
Dans un conte américain, huit petits canetons s’appellent respectivement :
Jack, Kack, Lack, Mack, Nack, Oack, Pack et Qack. Écrivez un petit script qui
génère tous ces noms à partir des deux chaînes suivantes :
\texttt{prefixes = 'JKLMNOP'} et \texttt{suffixe = 'ack'}
Si vous utilisez une instruction \texttt{for ... in ...}, votre script ne devrait
comporter que deux lignes.
\end{exercice}

\begin{exercice}
Écrire une boucle qui affiche ceci:
{\footnotesize
\begin{verbatim}
+++++
++++++++
+++++++++++
++++++++++++++
+++++++++++++++++
++++++++++++++++++++
+++++++++++++++++++++++
\end{verbatim}
}
\end{exercice}

\begin{exercice}
    Une intégrale peut se calculer comme la limite d'une somme de Riemann:
    \[
	\int_0^1 x^2\,dx =
	\lim_{n\to\infty}\frac{1}{n}\sum_{i=0}^n{\left(\frac{i}{n}\right)^2}
    \]
    Écrire une boucle qui calcule la somme de Riemann ci-dessus pour les
    valeurs de $n=10, 100, 1000, 10000$. Combien de chiffres après la virgule
    sont corrects? Comment ce nombre de chiffres évolue lorsque $n$ est
    multiplié par 10?
\end{exercice}

Pour les prochains exercices, vous pouvez utiliser SymPy.

\begin{exercice}
    Écrire une boucle qui affiche la factorisation de tous les entiers de 1 à
    100.
\end{exercice}

\begin{exercice}
    Écrire une boucle qui affiche la dérivée des fonctions $f_n(x)=x^n$ pour
    toutes les valeurs de $n$ de 1 à 20.
\end{exercice}

\begin{exercice}
    Écrire une boucle qui affiche la primitive des fonctions $\sin(x)$,
    $\cos(x)$, $\tan(x)$, $\log(x)$, $\exp(x)$, $\sinh(x)$ et $\cosh(x)$.
\end{exercice}

\begin{exercice}
    Définir une matrice carrée $A$ de votre choix. Écrire une boucle qui
    affiche les 10 premières puissances de la matrice $A$. Comment se
    comportent les coefficients?
\end{exercice}



