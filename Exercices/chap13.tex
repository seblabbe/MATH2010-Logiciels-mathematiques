
\section*{Exercices du Chapitre 13: Boucle \texttt{for}}

% Les exercices ci-bas sont tirés du livre de Gérard Swinnen 
% \footnote{Gérard Swinnen, Apprendre à programmer avec Python 3, 2012.
% \url{http://inforef.be/swi/download/apprendre_python3_5.pdf}}
% que vous êtes invités à consulter pour en apprendre d'avantage.

\begin{exercice}
Soient les listes suivantes :
\begin{verbatim}
    t1 = [31, 28, 31, 30, 31, 30, 31, 31, 30, 31, 30, 31]
    t2 = ['Janvier', 'Février', 'Mars', 'Avril', 'Mai', 'Juin',
    'Juillet', 'Août', 'Septembre', 'Octobre', 'Novembre', 'Décembre']
\end{verbatim}
En utilisant \texttt{t1} et \texttt{t2}
créer une nouvelle liste \texttt{t3} qui contient
tous les éléments des deux listes en les alternant, de telle manière
que chaque nom de mois soit suivi du nombre de jours correspondant :
\texttt{['Janvier',31,'Février',28,'Mars',31, etc...]}.
\end{exercice}

\begin{exercice}
Dans un conte américain, huit petits canetons s’appellent respectivement :
Jack, Kack, Lack, Mack, Nack, Oack, Pack et Qack. Écrivez un petit script qui
génère tous ces noms à partir des deux chaînes suivantes :
\texttt{prefixes = 'JKLMNOP'} et \texttt{suffixe = 'ack'}
Si vous utilisez une instruction \texttt{for ... in ...}, votre script ne devrait
comporter que deux lignes.
\end{exercice}

\begin{exercice}
Écrire une boucle qui affiche ceci:
%{\footnotesize
\begin{verbatim}
+++++
++++++++
+++++++++++
++++++++++++++
+++++++++++++++++
++++++++++++++++++++
+++++++++++++++++++++++
\end{verbatim}
%}
\end{exercice}

\begin{exercice}
Écrire une boucle qui recherche le plus grand élément présent dans une liste
donnée. Par exemple, si on l’appliquait à la liste \texttt{[32, 5, 12, 8, 3, 75,
2, 15]} ce programme devrait afficher : \texttt{le plus grand élément de cette liste a
la valeur 75}.
\end{exercice}

\begin{exercice}
Écrivez un programme qui analyse un par un tous les éléments d’une liste de
nombres (par exemple celle de l’exercice précédent) pour générer deux nouvelles
listes. L’une contiendra seulement les nombres pairs de la liste initiale, et
l’autre les nombres impairs. Par exemple, si la liste initiale est celle de
l’exercice précédent, le programme devra construire une liste pairs qui
contiendra \texttt{[32, 12, 8, 2]}, et une liste impairs qui contiendra
\texttt{[5, 3, 75, 15]}. Astuce : pensez à utiliser l’opérateur modulo (\%).
\end{exercice}

\begin{exercice}
Écrivez un programme qui compte le nombre de consonnes contenus dans une phrase
donnée.
\end{exercice}

\begin{exercice}
Écrivez un programme qui analyse un par un tous les éléments d’une liste
de mots (par exemple : \texttt{['Jean', 'Maximilien', 'Brigitte', 'Sonia',
'Jean-Pierre', 'Sandra']}) pour générer deux nouvelles listes. L’une contiendra
les mots comportant moins de 6 caractères, l’autre les mots comportant 6
caractères ou davantage.
\end{exercice}

