\section{Fonctions \texttt{def}}

\begin{question}
Écrire une fonction qui calcule le périmètre et l'aire d’un triangle quelconque
dont l'utilisateur fournit les 3 côtés.
(Rappel : l'aire d’un triangle quelconque se calcule à l'aide de la formule
$\sqrt{d(d-a)(d-b)(d-c)}$ dans laquelle $d$ désigne la longueur du
demi-périmètre, et $a$, $b$, $c$ celles des trois côtés.)
\end{question}

\begin{exercice}
La durée d'ensoleillement $D(\beta, d)$ à un lieu donné sur la terre est donné par
la formule
\[
D(\beta,d) = 24 - \frac{24}{\pi}\arccos\left( \tan \beta \cdot
\tan\left(\arcsin\left(\sin(\kappa)\cdot \sin\left(\frac{2\pi}{365}d
\right)\right)\right)\right)
\]
où $\kappa=\frac{23,44}{180}\pi$ est l'inclinaison de la terre en radians,
$d\in[0,365]$ est le nombre de jours après l'équinoxe du printemps et
$\beta\in[-\pi/2,\pi/2]$ est la lattitude du lieu considéré.
Écrire en Python et en important les fonctions nécessaires du module
\texttt{math} la fonction \texttt{D(beta, d)}. Construire la liste des durées
d'ensoleillement à Liège pour les 31 jours du mois de mai 2016.
% http://maths-au-quotidien.fr/lycee/duree.pdf
% >>> D = 24 - S(24)/pi*acos(tan(beta)*tan(asin(sin(kappa)*sin(pi*S(2)/365*d))))
% >>> DD = 24 - S(24)/pi*acos(tan(beta)*tan(alpha))
\end{exercice}

\begin{question}
Soit la suite $u_{n+1}= \frac{1}{1+u_n^2}$ avec $u_0=0$.
Écrire une fonction \texttt{U(n)} qui retourne la valeur de $u_n$. Calculer $u_{20}$.
\end{question}

\begin{question}
Écrire une fonction \texttt{produit\_des\_chiffres(n)} qui retourne le produit
des chiffres de $n$ écrit en base 10. 
\end{question}

