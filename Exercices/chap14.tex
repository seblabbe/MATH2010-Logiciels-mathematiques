
\section*{Exercices du Chapitre 14: Conditions \texttt{if}}

\begin{question}
Que fait le programme ci-dessous, dans les quatre cas où l'on aurait défini au
préalable que la variable \texttt{a} vaut 1, 2, 3 ou 15?
\begin{verbatim}
if a != 2: 
    print('perdu')
elif a == 3:
    print('un instant, s.v.p.')
else: 
    print('gagné')
\end{verbatim}
\end{question}

\begin{exercice}
Écrivez une boucle qui compte le nombre de consonnes contenus dans une chaîne de
caractères.
\end{exercice}

\begin{exercice}
Écrire une boucle qui recherche le plus grand élément présent dans une liste
donnée. Par exemple, si on l’appliquait à la liste \texttt{[32, 5, 12, 8, 3, 75,
2, 15]} ce programme devrait afficher : \texttt{le plus grand élément de cette liste a
la valeur 75}.
\end{exercice}

\begin{question}
Écrire une boucle qui analyse un par un tous les éléments d'une liste de nombres
pour générer deux nouvelles listes. L'une contiendra seulement les nombres pairs
de la liste initiale, et l'autre les nombres impairs.
\end{question}

\begin{exercice}
Écrire une boucle qui analyse un par un tous les éléments d’une liste de mots
(par exemple : \texttt{['Jean', 'Maximilien', 'Brigitte', 'Sonia',
'Jean-Pierre', 'Sandra']}) pour générer deux nouvelles listes. L’une contiendra
les mots comportant moins de 6 caractères, l’autre les mots comportant 6
caractères ou davantage.
\end{exercice}


% \section*{Exercices du Chapitre 15: Fonctions \texttt{def}}
% 
% \begin{question}
% Écrire une fonction qui calcule le périmètre et l'aire d’un triangle quelconque
% dont l'utilisateur fournit les 3 côtés.
% (Rappel : l'aire d’un triangle quelconque se calcule à l'aide de la formule
% $\sqrt{d(d-a)(d-b)(d-c)}$ dans laquelle $d$ désigne la longueur du
% demi-périmètre, et $a$, $b$, $c$ celles des trois côtés.)
% \end{question}
% 
% \begin{exercice}
% La durée d'ensoleillement $D(\beta, d)$ à un lieu donné sur la terre est donné par
% la formule
% \[
% D(\beta,d) = 24 - \frac{24}{\pi}\arccos\left( \tan \beta \cdot
% \tan\left(\arcsin\left(\sin(\kappa)\cdot \sin\left(\frac{2\pi}{365}d
% \right)\right)\right)\right)
% \]
% où $\kappa=\frac{23,44}{180}\pi$ est l'inclinaison de la terre en radians,
% $d\in[0,365]$ est le nombre de jours après l'équinoxe du printemps et
% $\beta\in[-\pi/2,\pi/2]$ est la lattitude du lieu considéré.
% Écrire en Python et en important les fonctions nécessaires du module
% \texttt{math} la fonction \texttt{D(beta, d)}. Construire la liste des durées
% d'ensoleillement à Liège pour les 31 jours du mois de mai 2016.
% % http://maths-au-quotidien.fr/lycee/duree.pdf
% % >>> D = 24 - S(24)/pi*acos(tan(beta)*tan(asin(sin(kappa)*sin(pi*S(2)/365*d))))
% % >>> DD = 24 - S(24)/pi*acos(tan(beta)*tan(alpha))
% \end{exercice}
% 
% \begin{question}
% Soit la suite $u_{n+1}= \frac{1}{1+u_n^2}$ avec $u_0=0$.
% Écrire une fonction \texttt{U(n)} qui retourne la valeur de $u_n$. Calculer $u_{20}$.
% \end{question}
% 
% \begin{question}
% Écrire une fonction \texttt{produit\_des\_chiffres(n)} qui retourne le produit
% des chiffres de $n$ écrit en base 10. 
% \end{question}
% 
% \section*{Exercices du Chapitre 16: Boucle \texttt{while}}
% 
% \begin{question}
% Écire une boucle \texttt{while} qui affiche les nombre de 0 à 20 en ordre
% croissant sans utiliser l'instruction \texttt{if}. Même question mais en ordre
% décroissant.
% \end{question}
% 
% \begin{question}
% Que fait le programme suivant?
% \begin{verbatim}
%     a, b, c = 1, 1, 1
%     while c < 11 :
%         print(c, ": ", b)
%         a, b, c = b, a+b, c+1
% \end{verbatim}
% \end{question}
% 
% \begin{question}
% En utilisant une boucle \texttt{while},
% écrire une fonction \texttt{orbite\_produit\_des\_chiffres(n)} qui retourne
% la liste des itérations successives de la fonction qui retourne le produit des
% chiffres:
% {\small
% \begin{verbatim}
% [n, produit_des_chiffres(n), produit_des_chiffres(produit_des_chiffres(n)), ..., z]
% \end{verbatim}}
% \noindent
% jusqu'à ce qu'un nombre calculé \texttt{z} s'écrive avec un seul chiffre.
% Pouvez-vous trouver un nombre $n$ dont la longueur de l'orbite est plus grande
% que 5? plus grande que 10?
% %Conjecture: $f^k(n)$ atteint un nombre < 10 en moins de k=11 iterations
% \end{question}
% 
% \begin{question}
% La série de Taylor de $\sin(x)$ est
% \[
%     \sin x= \lim_{n\to\infty}\sum^{n}_{k=0} \frac{(-1)^k}{(2k+1)!} x^{2k+1} = x -
% \frac{x^3}{3!} + \frac{x^5}{5!} - \cdots
% \]
% Écrire une fonction \texttt{taylor\_sin(x)} qui évalue la série de Taylor en
% négligeant les termes de la somme qui sont inférieurs à $10^{-5}$ en valeur absolue.
% \end{question}
% 
% \section*{Exercices du Chapitre 17: Programmation avec \texttt{def} + \texttt{while} +
% \texttt{for} + \texttt{if}}
% 
% \begin{question}
% Deux nombres premiers $p$ et $q$ avec $p<q$ sont jumeaux si $q-p=2$. Trouver
% tous les nombres premiers jumeaux inférieurs à 10000.
% \end{question}
% 
% \begin{question}
% Le nombre 6 est un nombre \emph{parfait}, car il est égal à la somme de ses
% diviseurs propres: $6=1+2+3$. Écrire une fonction \texttt{est\_parfait(n)} qui
% retourne vrai ou faux selon que le nombre \texttt{n} est parfait.
% Trouver les quatre nombres parfaits inférieurs à 10000. Trouver tous les nombres
% parfaits inférieurs à 1000000.
% \end{question}
% 
% 
% %\section*{Exercices du Chapitre 18: Autres structures de données}
% 
% 
% 
% % \begin{question}[Conjecture de Goldbach]
% % ~
% % \begin{enumerate}
% % \item
% % Écrire une fonction $\texttt{est\_premier}(n)$ qui renvoie \texttt{True} si $n$ est premier
% % et \texttt{False} sinon.
% % \item
% % Vérifier expérimentalement l'assertion suivante : « Pour tout nombre entier pair $n \geq 4$,
% % il existe deux nombres premier $p \geq 2$ et $q \geq 2$ tels que $n = p + q$. »
% % Jusqu'à quelle valeur de $n$ arrivez-vous vérifier la conjecture ?
% % \end{enumerate}
% % \end{question}
% 
% \begin{question}[Problème de Collatz]
% Étudier les itérations successives de la fonction $f : \N \rightarrow \N$ définie par
% $f(n) = 3n+1$ si $n$ est impair
% et $f(n) = n/2$ sinon.
% Qu'observez-vous ?
% \end{question}
% 
% % \begin{question}[Exponentielle]
% % On pose un grain de blé sur la première case d'un échiquier ($64$ cases),
% % puis deux sur la deuxième case, quatre sur la troisième case et $2^n$ sur la $n$ième case jusqu'à~$64$.
% % En supposant qu'un grain de riz pèse $4$ grammes, quel est la masse totale (en tonnes) du riz sur l'échiquier~?
% % (Note~: la production mondiale de riz par an est de $738$ millions de tonnes par an (2012).)
% % \end{question}
% 
% 
% % \begin{question}
% % Le nombre $2520$ est le plus petit nombre divisible par $1, 2, 3, \ldots, 10$.
% % Quel est le plus petit nombre divisible par $1, 2, 3, \ldots, 20$~?
% % \end{question}
% % 
% % 
% % \begin{question}
% % La somme des chiffres de $2^{15} = 32768$ est égale à $3 + 2 + 7 + 6 + 8 = 26$.
% % Quelle est la somme des chiffres de $2^{1000}$ ?
% % \end{question}
% 
% 
% \begin{question}[Série harmonique]
% La quantité
% $1 + \frac{1}{2} + \frac{1}{3} + \frac{1}{4} + \cdots$
% est-elle bornée ?
% Même question pour
% $1 + \frac{1}{2^2} + \frac{1}{3^2} + \frac{1}{4^2} + \cdots$
% \end{question}
% 
% 
% \begin{question}[$\sqrt{2}$ est irrationnel]
% Dire que $\sqrt{2}$ est irrationnel signifie qu'il n'existe aucun nombre rationnel $\frac{a}{b}$ tel que $\frac{a}{b} = \sqrt{2}$.
% Autrement dit, cela signifie il n'existe pas d'entiers strictement positifs $a$ et $b$ tels que $a^2 = 2b^2$.
% Vérifier cette affirmation expérimentalement.
% \end{question}
% 
% 
% \begin{question}
% Existe-t-il deux nombres entiers positifs $x$ et $y$ tels que $x^2 - 61y^2 = 1$ ?
% %1766319049,22615398
% \end{question}
% 
% 
% \begin{question}
% Combien y a-t-il de chaînes de caractères de longueur $n$,
% qui ne contiennent que les lettres \texttt{a} et \texttt{b}
% mais qui ne contiennent pas « \texttt{aa} » ?
% Calculer au moins jusqu'à $n = 10$.
% Reconnaissez-vous la suite qui apparaît ?
% \end{question}
% 
% 
% \begin{question}
% Trouver un nombre $n$ tel que $2013 \times n$ ne s'écrit qu'avec des « $1$ ».
% \end{question}


