
\section*{Divers}

\begin{exercice}
    Vérifier si la droite qui passe par les points $(0,1)$ et $(1,1)$ est
    tangente au cercle de centre $(5,5)$ et de rayon $3$.
%\verb|Circle(Point(5,5), 3).is_tangent(|\\\verb|Line(Point(0,1), Point(1,1)))|
\end{exercice}

\begin{exercice}
    TODO: BIG DATA. Mettre des ronds sur une carte.
\end{exercice}

\begin{exercice}
Refaire le premier graphique de l'Introduction du livre de Pikkety.
\url{http://piketty.pse.ens.fr/files/capital21c/xls/}
Lire l'introduction ici...
\end{exercice}

\begin{exercice}
    TODO: Hypothèse de Riemann (voir livre de W. Stein).
\end{exercice}

\begin{exercice}
La durée d'ensoleillement $D(\beta, d)$ à un lieu donné sur la terre est donné par
la formule
\[
D(\beta,d) = 24 - \frac{24}{\pi}\arccos\left( \tan \beta \cdot
\tan\left(\arcsin\left(\sin(\kappa)\cdot \sin\left(\frac{2\pi}{365}d
\right)\right)\right)\right)
\]
où $\kappa=\frac{23,44}{180}\pi$ est l'inclinaison de la terre en radians,
$d\in[0,365]$ est le nombre de jours après l'équinoxe du printemps et
$\beta\in[-\pi/2,\pi/2]$ est la lattitude du lieu considéré.

% http://maths-au-quotidien.fr/lycee/duree.pdf
% >>> D = 24 - S(24)/pi*acos(tan(beta)*tan(asin(sin(kappa)*sin(pi*S(2)/365*d))))
% >>> DD = 24 - S(24)/pi*acos(tan(beta)*tan(alpha))

    TODO: Courbe de la durée d'enseigement à Liège. Calculer la moyenne sur
    une période.
\end{exercice}

\section{deja fait}

\begin{exercice}
% from http://docs.sympy.org/dev/tutorial/gotchas.html
En exécutant le code ci-dessous en SymPy, qu'est-ce que la ligne
\texttt{print(expr)} imprime à l'écran ? Expliquer pourquoi.
\begin{verbatim}
   from sympy.abc import x
   expr = x + 1
   x = 2
   print(expr)
\end{verbatim}
Est-ce que Mathematica se comporte de la même façon?
\end{exercice}

\begin{exercice}
Est-ce que le code suivant fait bien ce qui est désiré? Expliquer pourquoi.
Si vous répondez non à la première question, comment le corriger?
\begin{verbatim}
>>> from sympy import factor
>>> from sympy.abc import x
>>> k = 5
>>> expr = x**k + 1
>>> factor(expr)
>>> k = 6
>>> factor(expr)
\end{verbatim}
\end{exercice}

% \begin{question}
% Soit un polygone régulier à $n$ côtés.
% Trouver l'expression symbolique $\Theta(n)$ qui donne la mesure (en radians)
% d'un angle intérieur du polygone en fonction de $n$.
% Tracer la fonction $\Theta(n)$ sur l'intervalle $[1,10]$ des valeurs de $n$.
% Déduire le comportement de $\Theta(n)$ lorsque $n$ tend vers l'infini.
% \end{question}

\begin{question}
    Une question qui demande de trouver l'erreur ou deviner ce qui sera
    imprimé en fonction d'un if, elif, boucles, etc.
\end{question}

\begin{question}
    Une question qui demande de simplifier des if imbriqués.
\end{question}

\begin{question}
Écrivez un programme qui analyse un par un tous les éléments d’une liste de
nombres pour générer deux nouvelles listes. L’une contiendra seulement les
nombres pairs de la liste initiale, et l’autre les nombres impairs.
\end{question}

\begin{question}
Écrivez un programme qui calcule le périmètre et l’aire d’un triangle quelconque
dont l’utilisateur fournit les 3 côtés.
(Rappel : l’aire d’un triangle quelconque se calcule à l’aide de la formule
$\sqrt{d(d-a)(d-b)(d-c)}$ dans laquelle $d$ désigne la longueur du
demi-périmètre, et $a$, $b$, $c$ celles des trois côtés.)
\end{question}

\begin{question}
Que fait le programme ci-dessous, dans les quatre cas où l'on aurait défini au
préalable que la variable \texttt{a} vaut 1, 2, 3 ou 15?
\begin{verbatim}
if a != 2: 
    print('perdu')
elif a == 3:
    print('un instant, s.v.p.')
else: 
    print('gagné')
\end{verbatim}
\end{question}

