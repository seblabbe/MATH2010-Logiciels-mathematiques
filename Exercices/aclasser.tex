
\section*{Divers}

\begin{exercice}
    TODO: BIG DATA. Mettre des ronds sur une carte.
\end{exercice}

\begin{exercice}
    Calculer la cardinalité de l'union de polytopes (polygones, ou ensembles)
    avec la formule inclusion-exclusion.
\end{exercice}

\begin{exercice}
    Vérifier si la droite qui passe par les points $(0,1)$ et $(1,1)$ est
    tangente au cercle de centre $(5,5)$ et de rayon $3$.
%\verb|Circle(Point(5,5), 3).is_tangent(|\\\verb|Line(Point(0,1), Point(1,1)))|
\end{exercice}

\section{deja fait}

% \begin{question}
% Soit un polygone régulier à $n$ côtés.
% Trouver l'expression symbolique $\Theta(n)$ qui donne la mesure (en radians)
% d'un angle intérieur du polygone en fonction de $n$.
% Tracer la fonction $\Theta(n)$ sur l'intervalle $[1,10]$ des valeurs de $n$.
% Déduire le comportement de $\Theta(n)$ lorsque $n$ tend vers l'infini.
% \end{question}

