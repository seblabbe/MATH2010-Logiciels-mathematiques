\section{Programmation avec \texttt{def} + \texttt{while} +
\texttt{for} + \texttt{if}}

\begin{question}
Deux nombres premiers $p$ et $q$ avec $p<q$ sont jumeaux si $q-p=2$. Trouver
tous les nombres premiers jumeaux inférieurs à 10000.
\end{question}

\begin{question}
Le nombre 6 est un nombre \emph{parfait}, car il est égal à la somme de ses
diviseurs propres: $6=1+2+3$. Écrire une fonction \texttt{est\_parfait(n)} qui
retourne vrai ou faux selon que le nombre \texttt{n} est parfait.
Trouver les quatre nombres parfaits inférieurs à 10000. Trouver tous les nombres
parfaits inférieurs à 1000000.
\end{question}



% \begin{question}[Conjecture de Goldbach]
% ~
% \begin{enumerate}
% \item
% Écrire une fonction $\texttt{est\_premier}(n)$ qui renvoie \texttt{True} si $n$ est premier
% et \texttt{False} sinon.
% \item
% Vérifier expérimentalement l'assertion suivante : « Pour tout nombre entier pair $n \geq 4$,
% il existe deux nombres premier $p \geq 2$ et $q \geq 2$ tels que $n = p + q$. »
% Jusqu'à quelle valeur de $n$ arrivez-vous vérifier la conjecture ?
% \end{enumerate}
% \end{question}

\begin{question}[Problème de Collatz]
Étudier les itérations successives de la fonction $f : \N \rightarrow \N$ définie par
$f(n) = 3n+1$ si $n$ est impair
et $f(n) = n/2$ sinon.
Jusqu'à quel nombre $n$ pouvez-vous vérifier que les itérations successives
sur un entier $n$ atteignent le nombre 1?
\end{question}

% \begin{question}[Exponentielle]
% On pose un grain de blé sur la première case d'un échiquier ($64$ cases),
% puis deux sur la deuxième case, quatre sur la troisième case et $2^n$ sur la $n$ième case jusqu'à~$64$.
% En supposant qu'un grain de riz pèse $4$ grammes, quel est la masse totale (en tonnes) du riz sur l'échiquier~?
% (Note~: la production mondiale de riz par an est de $738$ millions de tonnes par an (2012).)
% \end{question}


% \begin{question}
% Le nombre $2520$ est le plus petit nombre divisible par $1, 2, 3, \ldots, 10$.
% Quel est le plus petit nombre divisible par $1, 2, 3, \ldots, 20$~?
% \end{question}
% 
% 
% \begin{question}
% La somme des chiffres de $2^{15} = 32768$ est égale à $3 + 2 + 7 + 6 + 8 = 26$.
% Quelle est la somme des chiffres de $2^{1000}$ ?
% \end{question}


\begin{question}[Série harmonique]
La quantité
$1 + \frac{1}{2} + \frac{1}{3} + \frac{1}{4} + \cdots$
est-elle bornée ?
Même question pour
$1 + \frac{1}{2^2} + \frac{1}{3^2} + \frac{1}{4^2} + \cdots$
\end{question}


\begin{question}[$\sqrt{2}$ est irrationnel]
Dire que $\sqrt{2}$ est irrationnel signifie qu'il n'existe aucun nombre rationnel $\frac{a}{b}$ tel que $\frac{a}{b} = \sqrt{2}$.
Autrement dit, cela signifie il n'existe pas d'entiers strictement positifs $a$ et $b$ tels que $a^2 = 2b^2$.
Vérifier cette affirmation expérimentalement.
\end{question}


\begin{question}
Existe-t-il deux nombres entiers positifs $x$ et $y$ tels que $x^2 - 61y^2 = 1$ ?
%1766319049,22615398
\end{question}


\begin{question}
Combien y a-t-il de chaînes de caractères de longueur $n$,
qui ne contiennent que les lettres \texttt{a} et \texttt{b}
mais qui ne contiennent pas « \texttt{aa} » ?
Calculer au moins jusqu'à $n = 10$.
Reconnaissez-vous la suite qui apparaît ?
\end{question}


\begin{question}
Trouver un nombre $n$ tel que $2013 \times n$ ne s'écrit qu'avec des « $1$ ».
\end{question}

\begin{question}
    Résoudre les premiers problèmes de \url{https://projecteuler.net/archives}
\end{question}



