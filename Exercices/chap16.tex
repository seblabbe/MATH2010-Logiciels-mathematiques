\section*{Exercices du Chapitre 16: Boucle \texttt{while}}

\begin{question}
Écire une boucle \texttt{while} qui affiche les nombre de 0 à 20 en ordre
croissant sans utiliser l'instruction \texttt{if}. Même question mais en ordre
décroissant.
\end{question}

\begin{question}
Que fait le programme suivant?
\begin{verbatim}
    a, b, c = 1, 1, 1
    while c < 11 :
        print(c, ": ", b)
        a, b, c = b, a+b, c+1
\end{verbatim}
\end{question}

\begin{question}
En utilisant une boucle \texttt{while},
écrire une fonction \texttt{orbite\_produit\_des\_chiffres(n)} qui retourne
la liste des itérations successives de la fonction qui retourne le produit des
chiffres:
{\small
\begin{verbatim}
[n, produit_des_chiffres(n), produit_des_chiffres(produit_des_chiffres(n)), ..., z]
\end{verbatim}}
\noindent
jusqu'à ce qu'un nombre calculé \texttt{z} s'écrive avec un seul chiffre.
Pouvez-vous trouver un nombre $n$ dont la longueur de l'orbite est plus grande
que 5? plus grande que 10?
%Conjecture: $f^k(n)$ atteint un nombre < 10 en moins de k=11 iterations
\end{question}

\begin{question}
La série de Taylor de $\sin(x)$ est
\[
    \sin x= \lim_{n\to\infty}\sum^{n}_{k=0} \frac{(-1)^k}{(2k+1)!} x^{2k+1} = x -
\frac{x^3}{3!} + \frac{x^5}{5!} - \cdots
\]
Écrire une fonction \texttt{taylor\_sin(x)} qui évalue la série de Taylor en
négligeant les termes de la somme qui sont inférieurs à $10^{-5}$ en valeur absolue.
\end{question}
