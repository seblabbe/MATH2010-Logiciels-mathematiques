
\section*{Exercices du Chapitre 1: Introduction et interface Jupyter}

\begin{exercice}[Ouvrir et fermer Jupyter]
    Créer un dossier où seront sauvegardées les feuilles de calculs de Jupyter.
    Ouvrir Jupyter. Ceci ouvrira un nouvel onglet dans le navigateur web.
    Sélectionner le dossier que vous venez de créer.
    Créer une nouvelle feuille de calcul en appuyant sur le bouton \emph{New}.
    Faire un premier calcul dans la cellule à côté de "\texttt{In [1]:}".
    Écrire \texttt{3+8} par exemple. Évaluer le résultat en appuyant sur les
    deux touches \texttt{MAJ+ENTRÉE}. Sauvegarder la feuille de calcul avec
    \texttt{CTRL+S} ou en cliquant sur l'icône \emph{disquette}. Fermer la
    feuille de calcul en appuyant sur \texttt{File > Close and Halt}. Fermer
    Jupyter. Ouvrir Jupyter, retrouver la feuille de calcul déjà
    sauvegardée et l'ouvrir.
\end{exercice}

\begin{exercice}[Se familiariser avec Jupyter]
    Dans Jupyter, faire le \texttt{User Interface Tour} du menu \texttt{Help}.
\end{exercice}


