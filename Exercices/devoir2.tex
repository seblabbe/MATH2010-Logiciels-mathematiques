\documentclass[12pt,a4paper]{article}

\usepackage{amsmath}
\usepackage{amsfonts}
\usepackage{latexsym}
\usepackage{graphicx}
\usepackage{amssymb}
\usepackage{amsthm}
\usepackage[margin=2cm,includehead]{geometry}
\usepackage[T1]{fontenc}     % Hyphénation, copier-coller des mots accentués
\usepackage{lmodern}         % Polices vectorielles
\usepackage[french]{babel}   % Vocabulaire en francais
\usepackage[utf8]{inputenc}  % Codage UNICODE (UTF-8)
\usepackage{url}
\usepackage{color}
\usepackage{enumerate}
\usepackage[shortlabels]{enumitem} %pour commencer les enumerations a des nombres differents
\usepackage[pdftex,pagebackref]{hyperref}
\usepackage{cleveref}
\usepackage{titlesec}  % for using titleformat
\usepackage{rotating}

%\usepackage[disable]{todonotes}
\usepackage{todonotes}

% Page foot 
\usepackage{fancyhdr}
\pagestyle{fancy}
\lhead{\sffamily\bfseries\small Exercices pour les TP}
\rhead{\sffamily\bfseries\small MATH2010-1 Logiciels mathématiques}
%\renewcommand{\headrulewidth}{0.4pt}% remove header line
\renewcommand{\footrulewidth}{0pt}% default is 0pt
%\cfoot{}

% ceci immite le format par defaut des titres de section de la classe scrartcl
\titleformat{\section}
  {\normalfont\sffamily\large\bfseries\centering\color{black}}
  {\thesection}{0.5em}{}

% entier, réels, etc
\newcommand{\N}{\mathbb{N}}
\newcommand{\Z}{\mathbb{Z}}
\newcommand{\Q}{\mathbb{Q}}
\newcommand{\R}{\mathbb{R}}
\newcommand{\sympy}{\texttt{SymPy} }
\newcommand{\python}{\texttt{Python} }

% others
\newcommand{\bw}{\mathbf{w}}
\renewcommand{\a}{{\vec{a}}}

% Blocks
\theoremstyle{definition}
\newtheorem{exercice}{Exercice}[section]
\newtheorem{exemple}[exercice]{Exemple}
\newtheorem{remarque}[exercice]{Remarque}
\newtheorem{question}[exercice]{Question}




\begin{document}

\section*{Devoir 2}

Répondre aux questions suivantes. Vos réponses doivent être manuscrites
(rédigées à la main) et utiliser le langage \texttt{Python 3}.
%Inclure la démarche, les réponses aux questions et les justifications. 
À remettre en main propre ou le glisser sous la porte de mon bureau avant le
\textbf{lundi 9 mai à 16h59} 
en indiquant votre numéro de matricule et votre nom. Un travail remis en
retard perdra la moitié des points.

\begin{question}[5 pts]
    En Python, on peut représenter une matrice par une liste de listes. Par
    exemple, voici deux matrices 4 par 4:
    \begin{verbatim}
    X = [[56, 39,  3, 41],
         [23, 78, 11, 62],
         [61, 26, 65, 51],
         [80, 98,  9, 68]]
    Y = [[51, 52, 53, 15],
         [ 1, 71, 46, 31],
         [99,  7, 92, 12],
         [15, 43, 36, 51]]
    \end{verbatim}
    \vspace{-1em}
    Écrire une fonction \texttt{somme(A, B)} qui retourne la somme de deux
    matrices quelconques \texttt{A} et \texttt{B} de format $4\times 4$.
\end{question}

\begin{question}[5 pts]
    Écrire une fonction \texttt{produit(A, B)} qui retourne le produit de deux
    matrices quelconques \texttt{A} et \texttt{B} de format $4\times 4$.
\end{question}

\begin{question}[5 pts]
    Un \emph{triplet de nombres premiers} est un ensemble de trois nombres
    premiers de la forme $(p, p + 2, p + 6)$ ou $(p, p + 4, p + 6)$.
    Écrire une fonction \texttt{triplets\_nombres\_premiers(n)} qui retourne
    la liste des \texttt{n} plus petits triplets de nombres premiers.
\end{question}

\begin{question}[5 pts]
    Un triplet de Pythagore est un ensemble de trois nombres strictement
    positifs $(a,b,c)$ tels que 
    \[
	a^2+b^2=c^2.
    \]
    Écrire une fonction
    \texttt{triplets\_pythagore(n)} qui retourne la liste de tous les triplets
    de Pythagore $(a,b,c)$ tels que $a\leq b\leq c\leq n$.
\end{question}

\end{document}

