\documentclass[12pt,a4paper]{article}

\usepackage{amsmath}
\usepackage{amsfonts}
\usepackage{latexsym}
\usepackage{graphicx}
\usepackage{amssymb}
\usepackage{amsthm}
\usepackage[margin=2cm,includehead]{geometry}
\usepackage[T1]{fontenc}     % Hyphénation, copier-coller des mots accentués
\usepackage{lmodern}         % Polices vectorielles
\usepackage[french]{babel}   % Vocabulaire en francais
\usepackage[utf8]{inputenc}  % Codage UNICODE (UTF-8)
\usepackage{url}
\usepackage{color}
\usepackage{enumerate}
\usepackage[shortlabels]{enumitem} %pour commencer les enumerations a des nombres differents
\usepackage[pdftex,pagebackref]{hyperref}
\usepackage{cleveref}
\usepackage{titlesec}  % for using titleformat
\usepackage{rotating}

%\usepackage[disable]{todonotes}
\usepackage{todonotes}

% Page foot 
\usepackage{fancyhdr}
\pagestyle{fancy}
\lhead{\sffamily\bfseries\small Exercices pour les TP}
\rhead{\sffamily\bfseries\small MATH2010-1 Logiciels mathématiques}
%\renewcommand{\headrulewidth}{0.4pt}% remove header line
\renewcommand{\footrulewidth}{0pt}% default is 0pt
%\cfoot{}

% ceci immite le format par defaut des titres de section de la classe scrartcl
\titleformat{\section}
  {\normalfont\sffamily\large\bfseries\centering\color{black}}
  {\thesection}{0.5em}{}

% entier, réels, etc
\newcommand{\N}{\mathbb{N}}
\newcommand{\Z}{\mathbb{Z}}
\newcommand{\Q}{\mathbb{Q}}
\newcommand{\R}{\mathbb{R}}
\newcommand{\sympy}{\texttt{SymPy} }
\newcommand{\python}{\texttt{Python} }

% others
\newcommand{\bw}{\mathbf{w}}
\renewcommand{\a}{{\vec{a}}}

% Blocks
\theoremstyle{definition}
\newtheorem{exercice}{Exercice}[section]
\newtheorem{exemple}[exercice]{Exemple}
\newtheorem{remarque}[exercice]{Remarque}
\newtheorem{question}[exercice]{Question}


