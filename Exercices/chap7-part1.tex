
\section{Calcul différentiel et intégral}

Pour les exercices suivants, il sera nécessaire d'importer les fonctions
et symboles suivants:
\begin{verbatim}
from sympy import oo,log,pi,cos,sin,sqrt
from sympy import limit,summation,integrate,diff,dsolve
from sympy.abc import a,f,j,n,x,y,z,theta,rho
\end{verbatim}

\begin{exercice}
    Évaluer la limite de la suite $\sqrt[n]{n}$.
%\verb|limit(n**(1/n), n, oo)|
\end{exercice}

\begin{exercice}
    Évaluer la limite à gauche de la fonction $\frac{|x|}{x}$ en 0.
%\verb|limit(abs(x)/x, x, 0, dir='-')|
\end{exercice}

\begin{exercice}
    Calculer les limites suivantes: 
    $\lim_{x\to0^+}\frac{1}{x}$,\hspace{5pt}
    $\lim_{x\to0^-}\frac{1}{x}$,\hspace{5pt}
    $\lim_{x\to0}\frac{1}{x}$.
    Qu'en déduisez-vous?
\end{exercice}


\begin{exercice}
  \'Etudier la convergence de la suite $(x_j)_j$ d\'efinie par
\[
  x_j = \frac{j}{\ln(\pi)} \left(\sqrt{\ln^2(\pi)+\frac{1}{j}}-
\ln(\pi)\right).
\]
\end{exercice}

\begin{exercice}
Étudier la convergence de la suite $(x_j)_j$ de terme g\'en\'eral
\[
x_j= \sqrt{2 j^2 +2j-3} - \sqrt[4]{4 j^4 + 3j^3 +2 j +1}
\]
\end{exercice}

% Remarque : pour celui-ci, sympy n'arrive pas à calculer la limite si on
% écrit la racine quatrième en exposant ; on est obligé d'écrire
% sqrt(sqrt(4 * j**4 + 3 * j**3 +2 * j + 1))

\begin{exercice}
  Étudier la convergence de la suite $(x_j)_j$ de terme g\'en\'eral
  \[
  x_j=  \sqrt[j]{j^a}.
\]
où $a$ est un paramètre réel.
\end{exercice}

\begin{exercice}
Calculer les limites
\[
\lim_{x\to +\infty} \sqrt{x+\sqrt{x+\sqrt{x}}}- \sqrt{x}
\quad\text{et}\quad
  \lim_{x\to -\infty} \frac{\sqrt{|4x^3-x|}}{\sqrt{5-x^3}}.
\]
\end{exercice}

\begin{exercice}
    Évaluer la somme $\sum_{n=1}^{63} n$.
\end{exercice}

\begin{exercice}
    Évaluer la somme $\sum_{n=0}^{100} n^2$.
%\verb|summation(n**2, (n, 0, 100))|
\end{exercice}

\begin{exercice}
    Évaluer la série $\sum_{n=1}^{\infty} \frac{1}{n^2}$.
%\verb|summation(1/n**2, (n, 0, oo))|
\end{exercice}

\begin{exercice}
  \'Etudier la convergence de la s\'erie num\'erique suivante :
  \[
   \sum_{j=1}^\infty \frac{(-1)^j a^j}{2j},
  \]
pour toutes les valeurs possibles du nombre r\'eel $a$. Dans le cas o\`u
elle converge, donner sa valeur.
\end{exercice}

