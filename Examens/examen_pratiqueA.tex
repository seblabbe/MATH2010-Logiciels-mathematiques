\documentclass[12pt,a4paper]{article}

\usepackage{amsmath}
\usepackage{amsfonts}
\usepackage{latexsym}
\usepackage{graphicx}
\usepackage{amssymb}
\usepackage{amsthm}
\usepackage[margin=2cm,includehead]{geometry}
\usepackage[T1]{fontenc}     % Hyphénation, copier-coller des mots accentués
\usepackage{lmodern}         % Polices vectorielles
\usepackage[french]{babel}   % Vocabulaire en francais
\usepackage[utf8]{inputenc}  % Codage UNICODE (UTF-8)
\usepackage{url}
\usepackage{color}
\usepackage{enumerate}
\usepackage[shortlabels]{enumitem} %pour commencer les enumerations a des nombres differents
\usepackage[pdftex]{hyperref}
\usepackage{cleveref}
\usepackage{titlesec}  % for using titleformat
\usepackage{rotating}

%\usepackage[disable]{todonotes}
\usepackage{todonotes}

% Page foot 
\usepackage{fancyhdr}
\pagestyle{fancy}
\lhead{\sffamily\bfseries\small Exercices pour les TP}
\rhead{\sffamily\bfseries\small MATH2010-1 Logiciels mathématiques}
%\renewcommand{\headrulewidth}{0.4pt}% remove header line
\renewcommand{\footrulewidth}{0pt}% default is 0pt
%\cfoot{}

% ceci immite le format par defaut des titres de section de la classe scrartcl
\titleformat{\section}
  {\normalfont\sffamily\large\bfseries\centering\color{black}}
  {\thesection}{0.5em}{}

% entier, réels, etc
\newcommand{\N}{\mathbb{N}}
\newcommand{\Z}{\mathbb{Z}}
\newcommand{\Q}{\mathbb{Q}}
\newcommand{\R}{\mathbb{R}}
\newcommand{\sympy}{\texttt{SymPy} }
\newcommand{\python}{\texttt{Python} }

% others
\newcommand{\bw}{\mathbf{w}}
\renewcommand{\a}{{\vec{a}}}

% Blocks
\theoremstyle{definition}
\newtheorem{exercice}{Exercice}[section]
\newtheorem{exemple}[exercice]{Exemple}
\newtheorem{remarque}[exercice]{Remarque}
\newtheorem{question}[exercice]{Question}





\begin{document}

{\bfseries\sffamily
\begin{tabular}{p{5cm}p{6cm}r}
    Nom: & Prénom: & Matricule:
\end{tabular}}
\medskip

\begin{center}
  \normalfont\sffamily\large\bfseries\color{black}
    MATH2010-1 Logiciels mathématiques\\
    Examen final pratique -- groupe A -- 17 juin 2016
\end{center}

Répondre aux questions suivantes. 
Vous avez droit à vos notes, à internet, mais pas aux moyens de communication
de tout type (courriel, téléphone, etc.)
Vos réponses doivent être manuscrites
(rédigées à la main) et utiliser \texttt{Sympy} sous \texttt{Python 3} ou
\texttt{Mathematica}.
Vous devez écrire à la main la réponse ainsi que la démarche pour l'obtenir,
c'est-à-dire le code ou les calculs que vous avez utilisés.
Si vous utilisez Python, vous pouvez supposer que toutes les fonctions de
\verb|sympy| ont été importées et que les variables \verb|x| et \verb|n| ont
été définies:
\begin{verbatim}
    from sympy import *
    from sympy.abc import x, n
\end{verbatim}
Vous devez mentionner toutes les autres variables et fonctions que vous
importez.

\begin{question}[2 pts]
% from http://docs.sympy.org/dev/tutorial/gotchas.html
En exécutant le code ci-dessous en SymPy, qu'est-ce que la ligne
\texttt{print(expr)} imprime à l'écran ? Expliquer pourquoi.
\begin{verbatim}
    from sympy.abc import x
    expr = x + 1
    x = 2
    print(expr)
\end{verbatim}
\begin{mybox}
    Réponse:
    \begin{reponse}
	\verb|x+1|
    \end{reponse}\\[5pt]
    Explications:
    \begin{reponse}
	La variable \verb|expr| contient la somme du \emph{symbole} $x$
	et du nombre 1 et n'a pas de référence à la \emph{variable} \verb|x|.
	Modifier la \emph{variable} \verb|x| n'a donc pas d'effet sur le
	contenu de la variable \verb|expr|.
    \end{reponse}
\end{mybox}
Est-ce que Mathematica se comporte de la même façon?
\begin{mybox}
    Réponse:\\[10pt]
    \begin{reponse}
	Non. Dans Mathematica, il n'y a pas de distinction entre
	une \emph{variable} et un \emph{symbole}.
    \end{reponse}
\end{mybox}
\end{question}

\begin{question}[1 pts]
    Écrire l'expression suivante sous un dénominateur commun et factoriser ce
    dénominateur:
\[
    \frac{13}{1764 \, {\left(x + 5\right)}} - \frac{1}{36 \, {\left(x -
    1\right)}} + \frac{1}{49 \, {\left(x - 2\right)}} + \frac{1}{42 \,
    {\left(x + 5\right)}^{2}}.
\]
\begin{mybox}
Démarche et réponse:
\begin{reponse}
\begin{verbatim}
>>> from sympy.abc import x
>>> from sympy import ratsimp
>>> A = 13/(1764*(x+5)) + 1/(42*(x+5)**2) - 1/(36*(x-1)) + 1/(49*(x-2))
>>> ratsimp(A).factor()
           1
------------------------
                       2
(x - 2)*(x - 1)*(x + 5)
\end{verbatim}
\end{reponse}
\end{mybox}
\end{question}



\begin{question}[1 pts]
Décomposer l'expression suivante en somme de fractions rationnelles:
\[
    f(x) = \frac{11 \, x^{4} - 143 \, x^{3} + 249 \, x^{2} + 2191 \, x -
    6790}{x^{5} - 22 \, x^{4} + 147 \, x^{3} - 166 \, x^{2} - 1376 \, x +
    3360}.
\]
\begin{mybox}
Démarche et réponse:
\begin{reponse}
\begin{verbatim}
>>> from sympy.abc import x
>>> numerateur = 11*x**4 - 143*x**3 + 249*x**2 + 2191*x - 6790
>>> denominateur = x**5 - 22*x**4 + 147*x**3 - 166*x**2 - 1376*x + 3360
>>> f = numerateur / denominateur
>>> f.apart()
    1        3         7       5
- ----- - -------- + ----- + ------
  x + 3          2   x - 7   x - 10
          (x - 4)
\end{verbatim}
\end{reponse}
\end{mybox}
\end{question}


\begin{question}[1 pts]
Calculer $\int_{-1}^1 f(x) \,dx$ pour la fonction rationnelle $f(x)$ définie à
la question précédente.
\begin{mybox}
Démarche et réponse:
\begin{reponse}
\begin{verbatim}
>>> from sympy import integrate
>>> integrate(f, (x,-1,1))
-7*log(8) - 5*log(11) - log(4) - 2/5 + log(2) + 5*log(9) + 7*log(6)
\end{verbatim}
\end{reponse}
\end{mybox}
\end{question}


\begin{question}[2 pts]
    Calculer la série suivante, en fonction de $x$:
    \[
	\sum_{n=0}^{\infty} {(n+2) x^{n+3}\over n!}
    \]
\begin{mybox}
Démarche et réponse:
\begin{reponse}
\begin{verbatim}
>>> from sympy.abc import x, n
>>> from sympy import factorial, oo, summation
>>> summation((n+2)*x**n/factorial(n), (n, 0, oo))
   x      x
x*e  + 2*e
\end{verbatim}
\end{reponse}
\end{mybox}
\end{question}

\newpage
\begin{question}[2 pts]
    Le nombre 2520 est le plus petit nombre divisible par
    $1, 2, 3, \dots, 10$.
    Quelle est le plus petit nombre divisible par 
    $1, 2, 3, \dots, 20$?
\begin{mybox}
Démarche et réponse:\\
\begin{reponse}
\begin{minipage}[t]{.40\textwidth}
\begin{verbatim}
>>> from sympy import lcm
>>> lcm(range(1,11))
2520
>>> lcm(range(1,21))
232792560
\end{verbatim}
\end{minipage}
\begin{minipage}[t]{0.5\textwidth}
\begin{verbatim}
>>> def plus_petit_commun_multiple(x, y):
...     multiple_de_x = x
...     while not multiple_de_x % y == 0:
...         multiple_de_x += x
...     return multiple_de_x
>>> R = 1
>>> for a in range(1,21):
...     R = plus_petit_commun_multiple(R,a)
>>> R
232792560
\end{verbatim}
\end{minipage}
\end{reponse}
\end{mybox}
\end{question}

\begin{question}[1 pts]
    La somme des chiffres de $2^{15}=32768$ est égale à $3+2+7+6+8=26$.
    Quelle est la somme des chiffres de $2^{1000}$ ?
\begin{mybox}
Démarche et réponse:
\begin{reponse}
\begin{verbatim}
>>> L = [int(i) for i in str(2**15)]
>>> sum(L)
26
>>> L = [int(i) for i in str(2**1000)]
>>> sum(L)
1366
\end{verbatim}
\end{reponse}
\end{mybox}
\end{question}


\begin{question}[1 pts]
    Chercher l'ensemble des nombres complexes $z$ tels que $|z|=|z-1|$.
\begin{mybox}
Démarche et réponse:
\begin{reponse}
\begin{verbatim}
>>> from sympy import solve, I, symbols, Eq
>>> a,b = symbols('a b', real=True)
>>> z = a + b*I
>>> eq = Eq(abs(z), abs(z-1))
>>> eq
   _________      ___________________
  /  2    2      /  2          2
\/  a  + b   = \/  a  - 2*a + b  + 1
>>> solve(eq, (a,b))
[{a: 1/2}]
\end{verbatim}
\end{reponse}
\end{mybox}
\end{question}




\newpage
\begin{question}[1 pts]
Trouver les racines complexes du polynôme suivant:
\[
    g(x) = x^5 - 7i x^4 + 9x^3 - 9i x^2 - 10x + 16i.
\]
\begin{mybox}
Démarche et réponse:
\begin{reponse}
\begin{verbatim}
>>> from sympy import I
>>> from sympy.abc import x
>>> B = x**5 - 7*I*x**4 + 9*x**3 - 9*I*x**2 - 10*x + 16*I
>>> sympy.factor(B, gaussian=True)
(x - 1)*(x + 1)*(x - 8*I)*(x - I)*(x + 2*I)
\end{verbatim}
\end{reponse}
\end{mybox}
\end{question}


\begin{question}[2 pts]
Résoudre le système d'équations linéaires suivant:
\[
\left\{
\begin{array}{r}
 x +  y +  z + 2w =5\\
3x + 5y      + 3w =4\\
 x + 4y +  z + 3w =1\\
 x + 4y      + 2w =1
\end{array}
\right.
\]
\begin{mybox}
Démarche et réponse:
\begin{reponse}
\begin{verbatim}
>>> from sympy import Matrix
>>> M = Matrix([
[0, 1, 1, 2, 5],
[3, 5, 0, 3, 4],
[0, 4, 1, 3, 1],
[1, 4, 0, 2, 1]])
>>> M.rref()
(Matrix([
 [1, 0, 0, 0,    -2],
 [0, 1, 0, 0, -11/2],
 [0, 0, 1, 0, -29/2],
 [0, 0, 0, 1,  25/2]]), [0, 1, 2, 3])
\end{verbatim}
\end{reponse}
\end{mybox}
\end{question}



\begin{question}[2 pts]
La conchoïde de Sluze est une famille de courbes planaires étudiées en 1662
par René François Walter, baron de Sluze. Elles sont définies par l'équation
polaire
\[
    {\displaystyle r=\sec \theta +a\cos \theta \,}. 
\]
Choisissez trois valeurs réelles du paramètre $a$ et dessiner la courbe en
fonction de celles que vous avez choisies. Pour quelles valeurs de $a$ est-ce
que la courbe s'intersecte?
\begin{mybox}
    Démarche et réponse:
\begin{reponse}
\begin{verbatim}
>>> from sympy import cos, sin, sec, pi
>>> from sympy.abc import theta, a
>>> r = sec(theta) + a*cos(theta)
>>> x = (r*cos(theta)).simplify()
>>> y = (r*sin(theta)).simplify()
>>> x
a*cos(theta)**2 + 1
>>> y
a*sin(theta)*cos(theta) + tan(theta)
>>> from sympy.plotting import plot_parametric
>>> plot_parametric(x.subs(a, -1.5), y.subs(a, -1.5), (theta, -pi, pi), 
               ylim=(-2,2), xlim=(-2,2))
>>> plot_parametric(x.subs(a, -0.8), y.subs(a, -0.8), (theta, -pi, pi), 
               ylim=(-2,2), xlim=(-2,2))
>>> plot_parametric(x.subs(a, 0.8), y.subs(a, 0.8), (theta, -pi, pi), 
               ylim=(-2,2), xlim=(-2,2))
\end{verbatim}
\end{reponse}
\end{mybox}
\end{question}


\begin{question}[2 pts]
    Soit la matrice
    \[
    A = \left(\begin{array}{rrrr}
	    62 & -110 & -82 & 52 \\
	    19 & 33 & -12 & -38 \\
	    -3 & 0 & -23 & 6 \\
	    36 & -55 & -43 & 16
    \end{array}\right).
    \]
    Calculer le polynôme caractéristique et les valeurs propres de $A$.
    Calculer le vecteur propre à droite associé à la valeur propre dont le
    module est le plus grand.
\begin{mybox}
Démarche et réponse:
\begin{reponse}
\begin{verbatim}
>>> from sympy import Matrix
>>> from sympy import init_printing
>>> init_printing(pretty_print=True, use_unicode=False)
>>> M = Matrix([
[62, -110, -82,  52],
[19,   33, -12, -38],
[-3,    0, -23,   6],
[36,  -55, -43,  16]])
>>> M.eigenvals()
{-22: 1, -11: 1, 33: 1, 88: 1}
>>> M.eigenvects()
[(-22, 1, [[3/2]]), (-11, 1, [[2/3]]), (33, 1, [[2]]), (88, 1, [[2]])]
           [   ]              [   ]             [ ]             [ ]
           [1/2]              [2/3]             [1]             [0]
           [   ]              [   ]             [ ]             [ ]
           [3/2]              [1/3]             [0]             [0]
           [   ]              [   ]             [ ]             [ ]
           [ 1 ]              [ 1 ]             [1]             [1]
\end{verbatim}
\end{reponse}
\end{mybox}
\end{question}



\begin{question}[2 pts]
    Le 3 avril 2016, le Consortium international des journalistes d'enquête
    (ICIJ) a publié les premiers articles au sujet des \emph{Panama Papers}
    \footnote{Les \emph{Panama Papers} désignent la fuite de plus de 11,5
	millions de documents confidentiels issus du cabinet d'avocats
	panaméen Mossack Fonseca, détaillant des informations sur plus de 214
	000 sociétés établies dans des paradis fiscaux
    ainsi que les noms de ceux qui y sont associés.}.
    Au départ, seuls les journalistes travaillant sous l'égide de
    l'ICIJ ont pu fouiller les données, mais il s’agit d’une tâche titanesque.
    Le 9 mai 2016, le consortium a décidé de faire appel au public pour
    pousser la recherche encore plus loin en publiant une base de données
    contenant des informations provenant des Panama Papers (sans les
    courriels et autres données personnelles).
    %\footnote{
    %ICIJ releases database revealing thousands of secret offshore companies, 9
    %mai 2016.
%\url{https://panamapapers.icij.org/blog/20160509-offshore-database-release.html}}
Celle-ci contient cinq fichiers au format \texttt{.csv}:
\begin{verbatim}
   Addresses.csv  (25M)   Entities.csv        (102M)    Officers.csv  (41M) 
   all_edges.csv  (41M)   Intermediaries.csv  (3,5M)
\end{verbatim}
\vspace{-6pt}
    Répondre aux questions suivantes.
% www.math.ulg.ac.be/hidden_files/Entities.csv
% www.math.ulg.ac.be/hidden_files/Intermediaries.csv
% www.math.ulg.ac.be/hidden_files/Officers.csv
    \begin{itemize}
	\item En utilisant \texttt{pandas}, ouvrir le fichier
	    \url{http://www.math.ulg.ac.be/hidden_files/Officers.csv} dans un
	    tableau de données.
	\item Combien de lignes possède ce tableau?
	\item Filtrer ce tableau pour conserver que les lignes dont la
	    colonne \texttt{country\_codes} est relative à la Belgique
	    (ignorer le cas ou plus d'un pays apparaît).
	\item En utilisant la méthode \texttt{value\_counts} d'une série,
	      afficher le nom des cinq officiers qui apparaissent le plus
              souvent dans le tableau filtré.
    \end{itemize}
\begin{mybox}
Démarche et réponse:
\begin{reponse}
\begin{verbatim}
>>> import pandas as pd
>>> url = "http://www.math.ulg.ac.be/hidden_files/Officers.csv"
>>> df = pd.read_csv(url)



>>> len(df)
345594



>>> df_belgique = df[df.country_codes=='BEL']



>>> df_belgique.name.value_counts().head(5)
THE BEARER                                  36
MICHAEL GERHARD JOHANNES GLEISSNER           8
ALGONQUIN TRUST, PANAMA                      2
Franck Robert WYSTYRK                        2
Mr. Michael Bernard ATWOOD                   2
Name: name, dtype: int64
\end{verbatim}
\end{reponse}
\end{mybox}
\end{question}




\end{document}

