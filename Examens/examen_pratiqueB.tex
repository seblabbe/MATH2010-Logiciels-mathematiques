\documentclass[12pt,a4paper]{article}

\usepackage{amsmath}
\usepackage{amsfonts}
\usepackage{latexsym}
\usepackage{graphicx}
\usepackage{amssymb}
\usepackage{amsthm}
\usepackage[margin=2cm,includehead]{geometry}
\usepackage[T1]{fontenc}     % Hyphénation, copier-coller des mots accentués
\usepackage{lmodern}         % Polices vectorielles
\usepackage[french]{babel}   % Vocabulaire en francais
\usepackage[utf8]{inputenc}  % Codage UNICODE (UTF-8)
\usepackage{url}
\usepackage{color}
\usepackage{enumerate}
\usepackage[shortlabels]{enumitem} %pour commencer les enumerations a des nombres differents
\usepackage[pdftex,pagebackref]{hyperref}
\usepackage{cleveref}
\usepackage{titlesec}  % for using titleformat
\usepackage{rotating}

%\usepackage[disable]{todonotes}
\usepackage{todonotes}

% Page foot 
\usepackage{fancyhdr}
\pagestyle{fancy}
\lhead{\sffamily\bfseries\small Exercices pour les TP}
\rhead{\sffamily\bfseries\small MATH2010-1 Logiciels mathématiques}
%\renewcommand{\headrulewidth}{0.4pt}% remove header line
\renewcommand{\footrulewidth}{0pt}% default is 0pt
%\cfoot{}

% ceci immite le format par defaut des titres de section de la classe scrartcl
\titleformat{\section}
  {\normalfont\sffamily\large\bfseries\centering\color{black}}
  {\thesection}{0.5em}{}

% entier, réels, etc
\newcommand{\N}{\mathbb{N}}
\newcommand{\Z}{\mathbb{Z}}
\newcommand{\Q}{\mathbb{Q}}
\newcommand{\R}{\mathbb{R}}
\newcommand{\sympy}{\texttt{SymPy} }
\newcommand{\python}{\texttt{Python} }

% others
\newcommand{\bw}{\mathbf{w}}
\renewcommand{\a}{{\vec{a}}}

% Blocks
\theoremstyle{definition}
\newtheorem{exercice}{Exercice}[section]
\newtheorem{exemple}[exercice]{Exemple}
\newtheorem{remarque}[exercice]{Remarque}
\newtheorem{question}[exercice]{Question}





\begin{document}

{\bfseries\sffamily
\begin{tabular}{p{5cm}p{6cm}r}
    Nom: & Prénom: & Matricule:
\end{tabular}}
\medskip

\begin{center}
  \normalfont\sffamily\large\bfseries\color{black}
    MATH2010-1 Logiciels mathématiques\\
    Examen final pratique -- groupe B -- 17 juin 2016
\end{center}

Répondre aux questions suivantes. 
Vous avez droit à vos notes, à internet, mais pas aux moyens de communication
de tout type (courriel, téléphone, etc.)
Vos réponses doivent être manuscrites
(rédigées à la main) et utiliser \texttt{Sympy} sous \texttt{Python 3} ou
\texttt{Mathematica}.
Vous devez écrire à la main la réponse ainsi que la démarche pour l'obtenir,
c'est-à-dire le code ou les calculs que vous avez utilisés.
Si vous utilisez Python, vous pouvez supposer que toutes les fonctions de
\verb|sympy| ont été importées et que les variables \verb|x| et \verb|n| ont
été définies:
\begin{verbatim}
    from sympy import *
    from sympy.abc import x, n
\end{verbatim}
Vous devez mentionner toutes les autres variables et fonctions que vous
importez.

\begin{question}[2 pts]
% from http://docs.sympy.org/dev/tutorial/gotchas.html
En exécutant le code ci-dessous en SymPy, qu'est-ce que la ligne
\texttt{print(expr)} imprime à l'écran ? Expliquer pourquoi.
\begin{verbatim}
    from sympy import sin, pi
    from sympy.abc import x
    expr = sin(x) + 1
    x = pi
    print(expr)
\end{verbatim}
\begin{mybox}
    Réponse:
    \begin{reponse}
	\verb|sin(x)+1|
    \end{reponse}\\[5pt]
    Explications:
    \begin{reponse}
	La variable \verb|expr| contient la somme de l'expression $sin(x)$ qui
	contient le \emph{symbole} $x$
	et du nombre 1 et n'a pas de référence à la \emph{variable} \verb|x|.
	Modifier la \emph{variable} \verb|x| n'a donc pas d'effet sur le
	contenu de la variable \verb|expr|.
    \end{reponse}
\end{mybox}
Est-ce que Mathematica se comporte de la même façon?
\begin{mybox}
    Réponse:\\[10pt]
    \begin{reponse}
	Non. Dans Mathematica, il n'y a pas de distinction entre
	une \emph{variable} et un \emph{symbole}.
    \end{reponse}
\end{mybox}
\end{question}

\begin{question}[1 pts]
    Écrire l'expression suivante sous un dénominateur commun et factoriser ce
    dénominateur:
\[
-\frac{17}{2304 \, {\left(x + 3\right)}} - \frac{1}{64 \, {\left(x - 1\right)}}
+ \frac{1}{1280 \, {\left(x - 5\right)}} + \frac{1}{45 \, x} - \frac{1}{96 \,
{\left(x + 3\right)}^{2}}
\]
\begin{mybox}
Démarche et réponse:
\begin{reponse}
\begin{verbatim}
>>> from sympy.abc import x
>>> C = -17/(2304*(x+3)) - 1/(96*(x+3)**2) - 1/(64*(x-1)) 
        + 1/(1280*(x-5)) + 1/(45*x)
>>> C.ratsimp().factor()
            1
--------------------------
                         2
x*(x - 5)*(x - 1)*(x + 3)
\end{verbatim}
\end{reponse}
\end{mybox}
\end{question}



\begin{question}[1 pts]
    Décomposer l'expression suivante en somme de fractions rationnelles:
    \[
	f(x) = \frac{2 \, x^{4} - 19 \, x^{3} - 108 \, x^{2} + 1751 \, x - 4950}
	{x^{5} - 17 \, x^{4} + 81 \, x^{3} + 17 \, x^{2} - 826 \, x + 1176}
    \]
\begin{mybox}
Démarche et réponse:
\begin{reponse}
\begin{verbatim}
>>> from sympy.abc import x
>>> numerateur = 2*x**4 - 19*x**3 - 108*x**2 + 1751*x - 4950
>>> denominateur = x**5 - 17*x**4 + 81*x**3 + 17*x**2 - 826*x + 1176
>>> f = numerateur / denominateur
>>> f.apart()
    3       8       3        2
- ----- + ----- - ----- + --------
  x + 3   x - 2   x - 4          2
                          (x - 7)
\end{verbatim}
\end{reponse}
\end{mybox}
\end{question}


\begin{question}[1 pts]
Calculer $\int_{-1}^1 f(x) \,dx$ pour la fonction rationnelle $f(x)$ définie à
la question précédente.
\begin{mybox}
Démarche et réponse:
\begin{reponse}
\begin{verbatim}
>>> from sympy import integrate
>>> integrate(f, (x,-1,1))
-8*log(3) - 3*log(12) + 1/12 + 3*log(10)
\end{verbatim}
\end{reponse}
\end{mybox}
\end{question}


\begin{question}[2 pts]
    Calculer la série suivante, en fonction de $x$:
    \[
	\sum_{n=0}^{\infty} {(n+1)x^{2n+1}\over (2n+1)!}
    \]
\begin{mybox}
Démarche et réponse:
\begin{reponse}
\begin{verbatim}
>>> from sympy import summation, factorial, oo
>>> from sympy.abc import n,x
>>> summation((n+1)*x**(2*n+1)/factorial(2*n+1), (n,0,oo))
 3 /3*cosh(x)   3*sinh(x)\
x *|--------- - ---------|
   |     2           3   |
   \    x           x    /
-------------------------- + sinh(x)
            6
\end{verbatim}
\end{reponse}
\end{mybox}
\end{question}



\newpage
\begin{question}[2 pts]
    Le nombre 2520 est le plus petit nombre divisible par
    $1, 2, 3, \dots, 10$.
    Quelle est le plus petit nombre divisible par 
    $1, 2, 3, \dots, 20$?
\begin{mybox}
Démarche et réponse:\\
\begin{reponse}
\begin{minipage}[t]{.40\textwidth}
\begin{verbatim}
>>> from sympy import lcm
>>> lcm(range(1,11))
2520
>>> lcm(range(1,21))
232792560
\end{verbatim}
\end{minipage}
\begin{minipage}[t]{0.5\textwidth}
\begin{verbatim}
>>> def plus_petit_commun_multiple(x, y):
...     multiple_de_x = x
...     while not multiple_de_x % y == 0:
...         multiple_de_x += x
...     return multiple_de_x
>>> R = 1
>>> for a in range(1,21):
...     R = plus_petit_commun_multiple(R,a)
>>> R
232792560
\end{verbatim}
\end{minipage}
\end{reponse}
\end{mybox}
\end{question}



\begin{question}[1 pts]
    La somme des chiffres de $3^{15}=14348907$ est égale à 
    $1+4+3+4+8+9+0+7+3=36$.
    Quelle est la somme des chiffres de $3^{1000}$ ?
\begin{mybox}
Démarche et réponse:
\begin{reponse}
\begin{verbatim}
>>> L = [int(i) for i in str(3**15)]
>>> sum(L)
36
>>> L = [int(i) for i in str(3**1000)]
>>> sum(L)
2142
\end{verbatim}
\end{reponse}
\end{mybox}
\end{question}

\begin{question}[1 pts]
    Chercher l'ensemble des nombres complexes $z$ tels que $|z|=|z-3i|$.
\begin{mybox}
Démarche et réponse:
\begin{reponse}
\begin{verbatim}
>>> from sympy import solve, I, symbols, Eq
>>> a,b = symbols('a b', real=True)
>>> z = a + b*I
>>> eq = Eq(abs(z), abs(z-3*I))
>>> eq
   _________      ___________________
  /  2    2      /  2    2
\/  a  + b   = \/  a  + b  - 6*b + 9
>>> solve(eq, (a,b))
[{b: 3/2}]
\end{verbatim}
\end{reponse}
\end{mybox}
\end{question}


\newpage
\begin{question}[1 pts]
Trouver les racines complexes du polynôme suivant:
\[
	g(x) = x^{4} + \left(i - 5\right) \, x^{3} - \left(20 i + 23\right) \, x^{2} +
	\left(121 i + 45\right) \, x - 210 i + 126
    \]
\begin{mybox}
Démarche et réponse:
\begin{reponse}
\begin{verbatim}
>>> from sympy import I
>>> from sympy.abc import x
>>> A = x**4 - 5*x**3 + I*x**3 - 23*x**2 - 20*I*x**2 
...        + 45*x + 121*I*x + 126 - 210*I
>>> A.factor(gaussian=True)
(x - 7)*(x - 3)*(x - 2*I)*(x + 5 + 3*I)
\end{verbatim}
\end{reponse}
\end{mybox}
\end{question}


\begin{question}[2 pts]
Résoudre le système d'équations linéaires suivant:
\[
\left\{
\begin{array}{r}
3x + 2y      + 2w = 5\\
 x +  y + 5z + 5w = 3\\
2x      +  z +  w = 1\\
     4y + 3z + 2w = 4
\end{array}
\right.
\]
\begin{mybox}
Démarche et réponse:
\begin{reponse}
\begin{verbatim}
>>> from sympy import Matrix
>>> M = Matrix([
[3, 2, 0, 2, 5],
[1, 1, 5, 5, 3],
[2, 0, 1, 1, 1],
[0, 4, 3, 2, 4]])
>>> M.rref()
(Matrix([
 [1, 0, 0, 0,  1/3],
 [0, 1, 0, 0,    1],
 [0, 0, 1, 0, -2/3],
 [0, 0, 0, 1,    1]]), [0, 1, 2, 3])
\end{verbatim}
\end{reponse}
\end{mybox}
\end{question}



\begin{question}[2 pts]
Le folium de Descartes est une courbe algébrique mathématique en forme de nœud
de ruban étudiée par Descartes en 1638. On peut la définir avec les équations
paramétriques suivantes:
\[
{\displaystyle x={{3ap} \over
{1+p^{3}}},\,y={{3ap^{2}} \over {1+p^{3}}}}.
\]
Choisir une valeur de votre choix pour la constante $a$. Dessiner cette courbe
en trouvant un intervalle réel approprié (c'est-à-dire qui crée un beau
dessin) pour la variable $p$.
\begin{mybox}
    Démarche et réponse:
\begin{reponse}
\begin{verbatim}
>>> from sympy.abc import p
>>> a = 0.5
>>> x = 3*a*p / (1+p**3)
>>> y = 3*a*p**2 / (1+p**3)
>>> from sympy.plotting import plot_parametric
>>> plot_parametric(x, y, (p, -10, 10), ylim=(-3,2), xlim=(-3,2))
\end{verbatim}
\end{reponse}
\end{mybox}
\end{question}


\begin{question}[2 pts]
    Soit la matrice
    \[
	A = \left(\begin{array}{rrrr}
	-274 & 98 & 362 & 16 \\
	-480 & 180 & 590 & 20 \\
	-63 & 21 & 94 & 7 \\
	-675 & 225 & 900 & 45
	\end{array}\right).
    \]
    Calculer le polynôme caractéristique et les valeurs propres de $A$.
    Calculer le vecteur propre à droite associé à la valeur propre dont le
    module est le plus grand.
\begin{mybox}
Démarche et réponse:
\begin{reponse}
\begin{verbatim}
>>> from sympy import Matrix
>>> from sympy import init_printing
>>> init_printing(pretty_print=True, use_unicode=False)
>>> M = Matrix([[-274, 98, 362, 16],
 [-480, 180, 590, 20],
 [-63, 21, 94, 7],
 [-675, 225, 900, 45]])
>>> M.charpoly(x).as_expr()
 4       3        2
 x  - 45*x  - 700*x  + 37500*x - 270000
>>> M.eigenvals()
{-30: 1, 10: 1, 20: 1, 45: 1}
>>> M.eigenvects()
[(-30, 1, [[1/3]]), (10, 1, [[3]]), (20, 1, [[1/3]]), (45, 1, [[2/5]])]
           [   ]             [ ]             [   ]             [   ]
           [2/3]             [5]             [ 1 ]             [2/5]
           [   ]             [ ]             [   ]             [   ]
           [ 0 ]             [1]             [ 0 ]             [1/5]
           [   ]             [ ]             [   ]             [   ]
           [ 1 ]             [0]             [ 0 ]             [ 1 ]
\end{verbatim}
\end{reponse}
\end{mybox}
\end{question}


\newpage
\begin{question}[2 pts]
    Le 3 avril 2016, le Consortium international des journalistes d'enquête
    (ICIJ) a publié les premiers articles au sujet des \emph{Panama Papers}
    \footnote{Les \emph{Panama Papers} désignent la fuite de plus de 11,5
	millions de documents confidentiels issus du cabinet d'avocats
	panaméen Mossack Fonseca, détaillant des informations sur plus de 214
	000 sociétés établies dans des paradis fiscaux
    ainsi que les noms de ceux qui y sont associés.}.
    Au départ, seuls les journalistes travaillant sous l'égide de
    l'ICIJ ont pu fouiller les données, mais il s’agit d’une tâche titanesque.
    Le 9 mai 2016, le consortium a décidé de faire appel au public pour
    pousser la recherche encore plus loin en publiant une base de données
    contenant des informations provenant des Panama Papers (sans les
    courriels et autres données personnelles).
    %\footnote{
    %ICIJ releases database revealing thousands of secret offshore companies, 9
    %mai 2016.
%\url{https://panamapapers.icij.org/blog/20160509-offshore-database-release.html}}
Celle-ci contient cinq fichiers au format \texttt{.csv}:
\begin{verbatim}
   Addresses.csv  (25M)   Entities.csv        (102M)    Officers.csv  (41M) 
   all_edges.csv  (41M)   Intermediaries.csv  (3,5M)
\end{verbatim}
\vspace{-6pt}
    Répondre aux questions suivantes.
% www.math.ulg.ac.be/hidden_files/Entities.csv
% www.math.ulg.ac.be/hidden_files/Intermediaries.csv
% www.math.ulg.ac.be/hidden_files/Officers.csv
    \begin{itemize}
	\item En utilisant \texttt{pandas}, ouvrir le fichier
	    \url{www.math.ulg.ac.be/hidden_files/Intermediaries.csv} dans un
	    tableau de données.
	\item En utilisant la méthode \texttt{value\_counts} d'une série,
	    quels sont les cinq pays qui apparaissent le plus souvent dans
	    la colonne \texttt{countries} ?
	\item Filtrer ce tableau pour conserver que les lignes dont la
	    colonne \texttt{country\_codes} est relative à la Belgique
	    (ignorer le cas ou plus d'un pays apparaît).
	\item Combien de lignes possède ce tableau?
    \end{itemize}
\begin{mybox}
Démarche et réponse:
\begin{reponse}
\begin{verbatim}
>>> import pandas as pd
>>> url = "http://www.math.ulg.ac.be/hidden_files/Intermediaries.csv"
>>> df = pd.read_csv(url)



>>> df.countries.value_counts().head(5)
Hong Kong         4886
United Kingdom    1540
United States     1500
Taiwan            1323
Switzerland       1321
Name: countries, dtype: int64



>>> df_belgique = df[df.country_codes=='BEL']



>>> len(df_belgique)
51
\end{verbatim}
\end{reponse}
\end{mybox}
\end{question}




\end{document}

