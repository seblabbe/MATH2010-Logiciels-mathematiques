
\documentclass[12pt,a4paper]{article}

\usepackage{amsmath}
\usepackage{amsfonts}
\usepackage{latexsym}
\usepackage{graphicx}
\usepackage{amssymb}
\usepackage{amsthm}
\usepackage[margin=2cm,includehead]{geometry}
\usepackage[T1]{fontenc}     % Hyphénation, copier-coller des mots accentués
\usepackage{lmodern}         % Polices vectorielles
\usepackage[french]{babel}   % Vocabulaire en francais
\usepackage[utf8]{inputenc}  % Codage UNICODE (UTF-8)
\usepackage{url}
\usepackage{color}
\usepackage{enumerate}
\usepackage[shortlabels]{enumitem} %pour commencer les enumerations a des nombres differents
\usepackage[pdftex,pagebackref]{hyperref}
\usepackage{cleveref}
\usepackage{titlesec}  % for using titleformat
\usepackage{rotating}

%\usepackage[disable]{todonotes}
\usepackage{todonotes}

% Page foot 
\usepackage{fancyhdr}
\pagestyle{fancy}
\lhead{\sffamily\bfseries\small Exercices pour les TP}
\rhead{\sffamily\bfseries\small MATH2010-1 Logiciels mathématiques}
%\renewcommand{\headrulewidth}{0.4pt}% remove header line
\renewcommand{\footrulewidth}{0pt}% default is 0pt
%\cfoot{}

% ceci immite le format par defaut des titres de section de la classe scrartcl
\titleformat{\section}
  {\normalfont\sffamily\large\bfseries\centering\color{black}}
  {\thesection}{0.5em}{}

% entier, réels, etc
\newcommand{\N}{\mathbb{N}}
\newcommand{\Z}{\mathbb{Z}}
\newcommand{\Q}{\mathbb{Q}}
\newcommand{\R}{\mathbb{R}}
\newcommand{\sympy}{\texttt{SymPy} }
\newcommand{\python}{\texttt{Python} }

% others
\newcommand{\bw}{\mathbf{w}}
\renewcommand{\a}{{\vec{a}}}

% Blocks
\theoremstyle{definition}
\newtheorem{exercice}{Exercice}[section]
\newtheorem{exemple}[exercice]{Exemple}
\newtheorem{remarque}[exercice]{Remarque}
\newtheorem{question}[exercice]{Question}




\begin{document}

{\bfseries\sffamily
\begin{tabular}{p{5cm}p{6cm}r}
    Nom: & Prénom: & Matricule:
\end{tabular}}
\bigskip

\begin{center}
  \normalfont\sffamily\large\bfseries\color{black}
    MATH2010-1 Logiciels mathématiques\\
    Examen final écrit -- 16 juin 2016
\end{center}
\bigskip

Répondre aux questions suivantes. Vos réponses doivent être manuscrites
(rédigées à la main) et utiliser le langage \texttt{Python 3}.
Vous avez le droit d'utiliser un crayon et rien d'autre.

%Inclure la démarche, les réponses aux questions et les justifications. 

% Idées:
% \begin{question}[5 pts]
%     Une question qui demande de trouver l'erreur ou deviner ce qui sera
%     imprimé en fonction d'un if, elif, boucles, etc.
% \end{question}
% \begin{question}[5 pts]
%     Une question qui demande de simplifier des if imbriqués.
% \end{question}


% \begin{question}[5 pts]
% Est-ce que le code suivant fait bien ce qui est désiré? Expliquer pourquoi.
% Si vous répondez non à la première question, comment le corriger?
% \begin{verbatim}
%     from sympy import factor
%     from sympy.abc import x
%     k = 5
%     expr = x**k + 1
%     factor(expr)
%     k = 6
%     factor(expr)
% \end{verbatim}
% \end{question}

\begin{question}[5 pts]
Que sera-t-il imprimé à l'écran par l'exécution du code suivant?
\begin{verbatim}
    L = []
    for i in range(10):
        if i % 2 == 0:
            L.append(i)
        elif i % 3 == 0:
            L.append(i*10)
        elif i == 6:
            L.append(i*100) 
        if i == 8:
            L.append(i*1000) 
    print(L)     
\end{verbatim}
\begin{mybox}
    Réponse:\\[5pt]
\begin{reponse}
    [0, 2, 30, 4, 6, 8, 8000, 90]
\end{reponse}
\end{mybox}
\end{question}

\begin{question}[5 pts]
    En supposant que \verb|isprime(p)| retourne \verb|True| si \verb|p|
    est premier et \verb|False| sinon,
    trouver trois façons d'améliorer le code suivant:
\begin{verbatim}
def triplets_nombres_premiers(n):
    L = []
    p = 3
    while len(L) < n:
        if isprime(p) == True and isprime(p+6) == True and isprime(p+2) == True:
            L.append((p, p+2, p+6))
        elif isprime(p) == True and isprime(p+6) == True and isprime(p+4) == True:
            L.append((p, p+4, p+6))
        p += 1
    return L
\end{verbatim}
\begin{mybox}
    \vspace{3mm}
    Façon 1:
    \begin{reponse}
	Enlever les \verb|== True|.
    \end{reponse}\\[4mm]
    Façon 2:
    \begin{reponse}
	Tester une seule fois \verb|isprime(p) and isprime(p+6)|.
    \end{reponse}\\[4mm]
    Façon 3:
    \begin{reponse}
	Remplacer \verb|p += 1| par \verb|p += 2|.
    \end{reponse}\\
\end{mybox}
\end{question}

\newpage
\begin{question}[3 pts]
    Trouver un nom adéquat pour la fonction suivante. Écrire une courte phrase
    entre la paire de triple-guillemets qui décrit ce que la fonction fait.
\begin{verbatim}

def _____________________________(n):
    L = []
    for i in range(1, n):
        if n % i == 0:
            L.append(i)
    return L
\end{verbatim}
\begin{mybox}
Nom de la fonction:
\begin{reponse}
\verb|diviseurs_propres|
\end{reponse}\\
Description:
\begin{reponse}
Retourne la liste des diviseurs propres de \verb|n|.
\end{reponse}
\end{mybox}
\end{question}

\begin{question}[7 pts]
Un nombre est dit \emph{parfait} s'il est égal à la somme de ses diviseurs
propres. Par exemple, le nombre 6 est parfait, car les diviseurs propres de 6
sont 1, 2 et 3 et $6=1+2+3$. Le nombre 15 n'est pas parfait, car ses diviseurs
propres sont 1, 3 et 5 et $15\neq 1+3+5$. Écrivez une fonction Python appelée
\verb|nombres_parfait(n)| qui retourne la liste des nombres parfaits
strictement inférieurs à \verb|n|. Vous pouvez structurer votre programme en
définissant d'autres fonctions si nécessaire.
\begin{mybox}
    Réponse:
\begin{reponse}
\begin{verbatim}
def nombres_parfaits(n):
    L = []
    for i in range(1, n):
        if sum(diviseurs_propres(i)) == i:
            L.append(i)
    return L
\end{verbatim}
\end{reponse}
\vspace{10cm}
\phantom{x}
\end{mybox}
\end{question}

\end{document}
